\documentclass{beamer}

\usepackage{bibentry}


\usepackage{natbib}
\setcitestyle{square, comma, numbers,sort&compress, super}
\usepackage{apalike}


\usepackage{color}				% Controle das cores
\usepackage{amsmath}

\usepackage{breqn}
\usepackage{siunitx}
\usepackage{tikz}
\usetikzlibrary{automata,petri,calc,positioning,shapes.geometric,arrows,shapes,backgrounds}


\usepackage[ruled]{algorithm}
\usepackage{algorithmic}
\usepackage{graphicx}

\usepackage[portuges]{babel}    
\usepackage[utf8]{inputenc}

\usepackage{tikz}
\usetikzlibrary{arrows,automata}
\usepackage{subfig}
\usepackage{pdflscape}
\usepackage{pgfgantt}


\usepackage{pdfpages}
\usepackage{siunitx}
\usepackage{multirow}
\usepackage{color}

\newcommand*\rot{\rotatebox{90}}
\bibstyle{apa}

\pgfdeclarelayer{background}
\pgfsetlayers{background,main}

\lecture{Uma Abordagem para Reconhecimento de Emoção por Expressão Facial baseada em Redes Neurais de Convolução}{lecture-text}

%\date{October $31^{th}$, 2017}
\date{Manaus, 30 de Julho de 2018}

% Copyright 2007 by Till Tantau
%
% This file may be distributed and/or modified
%
% 1. under the LaTeX Project Public License and/or
% 2. under the GNU Public License.
%
% See the file doc/licenses/LICENSE for more details.


% Common packages

\usepackage{times}
\mode<article>
{
  \usepackage{times}
  \usepackage{mathptmx}
  \usepackage[left=1.5cm,right=6cm,top=1.5cm,bottom=3cm]{geometry}
}

\usepackage{hyperref}
\usepackage[T1]{fontenc}
\usepackage{tikz}
\usepackage{colortbl}
\usepackage{yfonts}
\usepackage{colortbl}
%\usepackage{translator} % comment this, if not available


% Common settings for all lectures in this course

\def\lecturename{XXVIII Simp\'{o}sio Brasileiro de Inform\'{a}tica na Educa\c{c}\~{a}o}

\title{\insertlecture}

\author{Anderson Cruz, Juan Colonna, Gabriel Leit\~{a}o, \\ Edson Silva, Raimundo Barreto, Tiago Primo} 

\institute
{

\begin{tiny}
 Instituto de Computa\c{c}\~ao (IComp), Centro de Desenvolvimento Tecnol\'{o}gico (CDTec)\\
 %Programa de P\'{o}s-Gradua\c{c}\~{a}o em Inform\'{a}tica, Programa de P\'{o}s-Gradua\c{c}\~{a}o em Computa\c{c}\~ao \\
 Universidade Federal do Amazonas (UFAM), Universidade Federal de Pelotas (UFPEL) \\
 Email: <\{aac, juancolonna, gabriel.leitao, rbarreto\}@icomp.ufam.edu.br, \\ edsonaraujo@ufam.edu.br, tiagoprimo@gmail.com > \\
\end{tiny}
}

\subject{\lecturename}




% Beamer version theme settings
\definecolor{VerdeUFAM}{rgb}{0.06,0.41,0.20}
\useoutertheme[height=0pt,width=2cm,left]{sidebar}
\usecolortheme{rose,sidebartab}
\useinnertheme{circles}
\usefonttheme[only large]{structurebold}

\setbeamercolor{sidebar left}{bg=black!15}
\setbeamercolor{structure}{fg=VerdeUFAM}
\setbeamercolor{author}{parent=structure}

\setbeamerfont{title}{series=\normalfont,size=\LARGE}
\setbeamerfont{title in sidebar}{series=\bfseries}
\setbeamerfont{author in sidebar}{series=\bfseries}
\setbeamerfont*{item}{series=}
\setbeamerfont{frametitle}{size=}
\setbeamerfont{block title}{size=\small}
\setbeamerfont{subtitle}{size=\normalsize,series=\normalfont}

\setbeamertemplate{navigation symbols}{}
\setbeamertemplate{bibliography item}[book]
\setbeamertemplate{sidebar left}
{
  \hbox to2cm{\hss\insertlogo\hss}
%  \vskip1.25em%
%   {\usebeamerfont{title in sidebar}%
% %    \vskip1.0em
%     \hskip1pt
%     \usebeamercolor[fg]{title in sidebar}%
%     \insertshorttitle[width=2cm,left,respectlinebreaks]\par%
%     \vskip1.25em%
%   }%
  \insertverticalnavigation{2cm}%
%  \vfill
  \hbox to 2cm{\hfill\usebeamerfont{subsection in
      sidebar}\strut\usebeamercolor[fg]{subsection in
      sidebar}}
  \vskip1pt%
  {%
    \hskip3pt%
    \usebeamercolor[fg]{author in sidebar}%
    \usebeamerfont{author in sidebar}%
    \insertshortauthor[width=2cm,center,respectlinebreaks]\par%
     \vskip4pt%
  }%
  \hbox to 2cm{
  \usebeamercolor[fg]{author in sidebar}%
    \usebeamerfont{author in sidebar}%
\insertframenumber\hskip5pt}%
}%

\setbeamertemplate{title page}
{
  \vbox{}
  \vskip1em
  {\huge \par}
  {\usebeamercolor[fg]{title}\usebeamerfont{title}\inserttitle\par}%
  \ifx\insertsubtitle\@empty%
  \else%
    \vskip0.25em%
    {\usebeamerfont{subtitle}\usebeamercolor[fg]{subtitle}\insertsubtitle\par}%
  \fi%     
  \vskip1em\par
  \emph{\lecturename}\par On: \insertdate\par
  \vskip0pt plus1filll
  \leftskip=0pt plus1fill\insertauthor\par
  \insertinstitute\vskip1em
}

\logo{\includegraphics[width=2cm]{figuras/brasaocor.jpg}}

% Typesetting Listings

\usepackage{listings}
\lstset{language=Java}

\alt<presentation>
{\lstset{%
  basicstyle=\footnotesize\ttfamily,
  commentstyle=\slshape\color{green!50!black},
  keywordstyle=\bfseries\color{blue!50!black},
  identifierstyle=\color{blue},
  stringstyle=\color{orange},
  escapechar=\#,
  emphstyle=\color{red}}
}
{
  \lstset{%
    basicstyle=\ttfamily,
    keywordstyle=\bfseries,
    commentstyle=\itshape,
    escapechar=\#,
    emphstyle=\bfseries\color{red}
  }
}



% Common theorem-like environments

\theoremstyle{definition}
\newtheorem{exercise}[theorem]{\translate{Exercise}}




% New useful definitions:

\newbox\mytempbox
\newdimen\mytempdimen

\newcommand\includegraphicscopyright[3][]{%
  \leavevmode\vbox{\vskip3pt\raggedright\setbox\mytempbox=\hbox{\includegraphics[#1]{#2}}%
    \mytempdimen=\wd\mytempbox\box\mytempbox\par\vskip1pt%
    \fontsize{3}{3.5}\selectfont{\color{black!25}{\vbox{\hsize=\mytempdimen#3}}}\vskip3pt%
}}

\newenvironment{colortabular}[1]{\medskip\rowcolors[]{1}{blue!20}{blue!10}\tabular{#1}\rowcolor{blue!40}}{\endtabular\medskip}

\def\equad{\leavevmode\hbox{}\quad}

\newenvironment{greencolortabular}[1]
{\medskip\rowcolors[]{1}{green!50!black!20}{green!50!black!10}%
  \tabular{#1}\rowcolor{green!50!black!40}}%
{\endtabular\medskip}




\setbeamertemplate{bibliography item}{\insertbiblabel}
\setbeamercovered{transparent}

\begin{document}

\begin{frame}
 \maketitle
\end{frame}

\begin{frame}{Agenda}
  \tableofcontents
\end{frame}

\section{Introdu\c{c}\~ao}
%\section{Motiva\c{c}\~ao}

%\begin{frame}
%\frametitle{Motiva\c{c}\~ao}
%\begin{itemize}
%\item Diversos psic\'ologos e pedagogos acreditam que o processo de cogni\c{c}\~ao n\~ao pode ser separado do afetivo:
%	\begin{itemize}	
%	\item \cite{vygotsky1986}: prop\~oe uma uni\~ao entre os processos intelectuais, evolutivos e afetivos, e  separa\c{c}\~ao gera lacunas;
%	\item \cite{piaget1989}: afetividade alimenta a atividade intelectual, isto \'e, n\~ao h\'a processo cognitivo sem afetivo;
%	\item \cite{izard1984}: emo\c{c}\~oes negativas prejudicam o desempenho do estudante em tarefas cognitivas e emo\c{c}\~oes 
%	positivas possuem um efeito contr\'ario;	
%	\end{itemize}
%\end{itemize}
%\end{frame}

\begin{frame}
\frametitle{Motiva\c{c}\~ao}
\pause

\cite{jaques2012}: A integra\c{c}\~ao de um reconhecedor de emo\c{c}\~oes com STI pode auxiliar grandes desafios da \'area de IE, 
viabilizando recomenda\c{c}\~oes mais inteligente, personalizada e prop\'icia para o estado emocional do estudante.

\end{frame}


\begin{frame}
\frametitle{Objetivos}
\begin{block}{Objetivos do trabalho:}
\begin{itemize}
\pause
\item Propor uma arquitetura de detec\c{c}\~ao autom\'atica de emo\c{c}\~oes para AVA por meio de reconhecimento autom\'atico de express\~oes faciais;
\pause
\item Analisar por meio de estat\'istica descritiva um estudo de caso com dados obtidos a partir da arquitetura, 
medindo correla\c{c}\~oes entre as emo\c{c}\~oes detectadas e o desempenho obtido nas quest\~oes.
\end{itemize}
\end{block}
\end{frame}


\section{Conceitos}
\begin{frame}
\frametitle{Conceitos}
\pause
\begin{block}{Express\~ao Facial Emocional}
\cite{ekman1994} verificou a exist\^encia de um grupo de emo\c{c}\~oes b\'asicas: raiva, felicidade, tristeza, desprezo, medo e surpresa, 
que possuem a mesma express\~ao facial independente da cultura dos indiv\'iduos.
\end{block}

\begin{figure}
\centering
\includegraphics[scale=0.27]{figuras/ekman-emotions.png}
\end{figure}

\end{frame}

\begin{frame}
\frametitle{Conceitos}
\begin{block}{Entropia}
Uma medida de dispers\~ao que seu resultado implica que quanto maior a entropia, maior \'e a uniformidade da distribui\c{c}\~ao dos dados. 
\end{block}
\begin{figure}
\centering
\includegraphics[scale=0.35]{figuras/entropiaSBIE.png}
\end{figure}
\end{frame}

\begin{frame}
\frametitle{Conceitos}
\begin{block}{Correla\c{c}\~ao de Pearson}
Mede o grau de correla\c{c}\~ao (for\c{c}a e dire\c{c}\~ao) entre um par de vari\'averis aleat\'orias.
\end{block}
\begin{figure}
\centering
\includegraphics[scale=0.4]{figuras/pearson.png}
\end{figure}

\end{frame}


\section{M\'etodo Proposto}
\begin{frame}
\frametitle{M\'etodo Proposto}
\pause
\begin{block}{Plataforma Educacional}
Permite a coleta de imagens faciais e das atividades dos alunos como a sele\c{c}\~ao de 
respostas e a navega\c{c}\~ao entre os diferentes objetos de aprendizagem.
\end{block}
\pause
\begin{block}{Classificador de emo\c{c}\~ao}
Recebe a imagem facial coletada como entrada e retorna as probabilidades \textit{a posteriori} das emo\c{c}\~oes b\'asicas.
\end{block}
\pause
\begin{block}{Anal\'{i}tico}
Recebe os dados da \textit{Plataforma Educacional} e do \textit{Classificador de Emo\c{c}\~ao} para processar e gerar informa\c{c}\~oes que s\~{a}o 
\'{u}teis para tutores inteligente, sistemas de recomenda\c{c}\~ao e heur\'isticas em geral.    
\end{block}

\end{frame}



\begin{frame}
\frametitle{M\'etodo Proposto}
\begin{figure}
\centering
\includegraphics[scale=0.35]{figuras/diagrama.png}
\end{figure}
\end{frame}

\section{Experimento}
\begin{frame}
\frametitle{Experimento}
\begin{itemize}
\pause
\item Consistiu em um simulado do ENEM para alunos finalistas do ensino m\'{e}dio;
\pause
\item Participaram 27 alunos;
\pause
\item O simulado era composto por 40 quest\~{o}es de m\'{u}ltipla escolha com 2 n\'{i}veis de dificuldade;
\pause
\item Os assuntos: matem\'{a}tica, l\'{i}ngua portuguesa, qu\'{i}mica, racioc\'{i}nio l\'{o}gico, geografia e hist\'{o}ria;
\pause
\item A \textit{Plataforma Educacional} escolhida foi resultado de um projeto de P\&D entre a UFAM e a Samsung;
\pause
\item O \textit{Classificador de Emo\c{c}\~ao} foi a API da \textit{Microsoft Cognitives Services} que classifica as 7 emo\c{c}\~oes b\'{a}sicas.
\end{itemize}

\end{frame}





%INICIO RESULTADOS E DISCUSSAO
\section{Resultados e Discuss\~oes}
\begin{frame}
\frametitle{Resultados e Discuss\~oes}
\pause
Foi realizado um pr\'{e}-processamento utilizando os dados da detec\c{c}\~ao das emoc\~oes e das intera\c{c}\~oes com a plataforma dos 27 alunos 
gerando para cada quest\~ao:

\begin{itemize}
\pause
\item A propor\c{c}\~ao de acertos;
\pause
\item O n\'{i}vel de dificuldade;
\pause
\item A m\'{e}dia das probabilidades para cada emoc\~ao detectada;
\pause
\item A entropia.
\end{itemize}

\end{frame}


\begin{frame}
\frametitle{Resultados e Discuss\~oes}
\begin{table}[]\footnotesize
\centering
\caption{Resultado​ ​da​ ​correla\c{c}\~ao​ ​de​ ​Pearson​ ​para​ ​cada​ ​emo\c{c}\~ao​ ​detectada
e​ ​a​ ​entropia​ ​contra​ ​os​ ​atributos​ ​das​ ​quest\~oes}
\label{my-label}
\begin{tabular}{|c|c|c|}
\hline
                      & \textbf{Nível de Dificuldade} & \textbf{Proporção de Acertos} \\ \hline
\textbf{Tristeza}     & \textbf{-0.33}                & 0.27                          \\ \hline
\textbf{Neutralidade} & \textbf{0.36}                 & \textbf{-0.48}                \\ \hline
\textbf{Desprezo}     & -0.15                         & \textbf{0.30}                 \\ \hline
Desgosto              & -0.13                         & 0.07                          \\ \hline
Raiva                 & -0.14                         & -0.08                         \\ \hline
Surpresa              & 0.07                          & 0.24                          \\ \hline
Medo                  & -0.06                         & 0.14                          \\ \hline
\textbf{Felicidade}   & -0.14                         & \textbf{0.31}                 \\ \hline
\textbf{Entropia}     & -0.12                         & \textbf{0.36}                 \\ \hline
\end{tabular}
\end{table}
\end{frame}


%\begin{frame}
%\frametitle{Resultados e Discuss\~oes}
%\begin{block}{Estimular emo\c{c}\~oes diferentes da neutralidade durante a avalia\c{c}\~ao favorece o desempenho dos alunos}
%\begin{itemize}
%\item Neutralidade possui uma correla\c{c}\~ao negativa com a propor\c{c}\~ao de acertos;
%\item Felicidade e desprezo possui correla\c{c}\~ao positiva com a propor\c{c}\~ao de acertos;
%\item Entropia das emo\c{c}\~oes possui correla\c{c}\~ao positiva com a propor\c{c}\~ao de acertos;
%\end{itemize}
%\end{block}

%\end{frame}

\begin{frame}
\frametitle{Resultados e Discuss\~oes}
\begin{block}{Estimular emo\c{c}\~oes diferentes da neutralidade durante a avalia\c{c}\~ao favorece o desempenho dos alunos}
\begin{itemize}
\pause
\item Neutralidade possui uma correla\c{c}\~ao negativa com a propor\c{c}\~ao de acertos;
\end{itemize}
\end{block}

\end{frame}

\begin{frame}
\frametitle{Resultados e Discuss\~oes}
\begin{table}[]\footnotesize
\centering
\caption{Resultado​ ​da​ ​correla\c{c}\~ao​ ​de​ ​Pearson​ ​para​ ​cada​ ​emo\c{c}\~ao​ ​detectada
e​ ​a​ ​entropia​ ​contra​ ​os​ ​atributos​ ​das​ ​quest\~oes}
\label{my-label}
\begin{tabular}{|c|c|c|}
\hline
                      & \textbf{Nível de Dificuldade} & \textbf{Proporção de Acertos} \\ \hline
Tristeza	     & -0.33                & 0.27                          \\ \hline
\small \textbf{Neutralidade} & 0.36                 & \small \textbf{-0.48}                \\ \hline
Desprezo     		& -0.15                         & 0.30                 \\ \hline
Desgosto              & -0.13                         & 0.07                          \\ \hline
Raiva                 & -0.14                         & -0.08                         \\ \hline
Surpresa              & 0.07                          & 0.24                          \\ \hline
Medo                  & -0.06                         & 0.14                          \\ \hline
Felicidade   		& -0.14                         & 0.31                 \\ \hline
Entropia     		& -0.12                         & 0.36                 \\ \hline
\end{tabular}
\end{table}
\end{frame}



\begin{frame}
\frametitle{Resultados e Discuss\~oes}
\begin{block}{Estimular emo\c{c}\~oes diferentes da neutralidade durante a avalia\c{c}\~ao favorece o desempenho dos alunos}
\begin{itemize}
\item Neutralidade possui uma correla\c{c}\~ao negativa com a propor\c{c}\~ao de acertos;
\pause
\item Felicidade e desprezo possui correla\c{c}\~ao positiva com a propor\c{c}\~ao de acertos.
\end{itemize}
\end{block}

\end{frame}

\begin{frame}
\frametitle{Resultados e Discuss\~oes}
\begin{table}[]\footnotesize
\centering
\caption{Resultado​ ​da​ ​correla\c{c}\~ao​ ​de​ ​Pearson​ ​para​ ​cada​ ​emo\c{c}\~ao​ ​detectada
e​ ​a​ ​entropia​ ​contra​ ​os​ ​atributos​ ​das​ ​quest\~oes}
\label{my-label}
\begin{tabular}{|c|c|c|}
\hline
                      & \textbf{Nível de Dificuldade} & \textbf{Proporção de Acertos} \\ \hline
Tristeza	     & -0.33                & 0.27                          \\ \hline
Neutralidade & 0.36                 & -0.48                \\ \hline
\small \textbf{Desprezo}     		& -0.15                         & \small \textbf{0.30}                 \\ \hline
Desgosto              & -0.13                         & 0.07                          \\ \hline
Raiva                 & -0.14                         & -0.08                         \\ \hline
Surpresa              & 0.07                          & 0.24                          \\ \hline
Medo                  & -0.06                         & 0.14                          \\ \hline
\small \textbf{Felicidade}   		& -0.14                         & \small \textbf{0.31}                 \\ \hline
Entropia     		& -0.12                         & 0.36                 \\ \hline
\end{tabular}
\end{table}
\end{frame}




\begin{frame}
\frametitle{Resultados e Discuss\~oes}
\begin{block}{Estimular emo\c{c}\~oes diferentes da neutralidade durante a avalia\c{c}\~ao favorece o desempenho dos alunos}
\begin{itemize}
\item Neutralidade possui uma correla\c{c}\~ao negativa com a propor\c{c}\~ao de acertos;
\item Felicidade e desprezo possui correla\c{c}\~ao positiva com a propor\c{c}\~ao de acertos;
\pause
\item Entropia das emo\c{c}\~oes possui correla\c{c}\~ao positiva com a propor\c{c}\~ao de acertos.
\end{itemize}
\end{block}

\end{frame}

\begin{frame}
\frametitle{Resultados e Discuss\~oes}
\begin{table}[]\footnotesize
\centering
\caption{Resultado​ ​da​ ​correla\c{c}\~ao​ ​de​ ​Pearson​ ​para​ ​cada​ ​emo\c{c}\~ao​ ​detectada
e​ ​a​ ​entropia​ ​contra​ ​os​ ​atributos​ ​das​ ​quest\~oes}
\label{my-label}
\begin{tabular}{|c|c|c|}
\hline
                      & \textbf{Nível de Dificuldade} & \textbf{Proporção de Acertos} \\ \hline
Tristeza	     & -0.33                & 0.27                          \\ \hline
Neutralidade & 0.36                 & -0.48                \\ \hline
Desprezo     		& -0.15                         & 0.30                 \\ \hline
Desgosto              & -0.13                         & 0.07                          \\ \hline
Raiva                 & -0.14                         & -0.08                         \\ \hline
Surpresa              & 0.07                          & 0.24                          \\ \hline
Medo                  & -0.06                         & 0.14                          \\ \hline
Felicidade   		& -0.14                         & 0.31                 \\ \hline
\small \textbf{Entropia}     		& -0.12                         & \small \textbf{0.36}                 \\ \hline
\end{tabular}
\end{table}
\end{frame}




%\begin{frame}
%\frametitle{Resultados e Discuss\~oes}
%\begin{block}{Neutralidade}
%\begin{itemize}
%\item A emo\c{c}\~ao mais frequente;
%\item Ind\'{i}cio da neutralidade assemelhar com estado de concentra\c{c}\~ao;
%\item Neutralidade possui uma correla\c{c}\~ao positiva com quest\~oes dif\'{i}ceis;
%\item Importante identificar os casos diferentes de neutralidade por serem menos frequentes 
%e indicar alguma oportunidade de um tutor inteligente atuar;
%\end{itemize}
%\end{block}

%\end{frame}








\begin{frame}
\frametitle{Resultados e Discuss\~oes}
\begin{block}{Neutralidade}
\begin{itemize}
\pause
\item A emo\c{c}\~ao mais frequente;
\pause
\item Ind\'{i}cio da neutralidade assemelhar com estado de concentra\c{c}\~ao;
\pause
\item Neutralidade possui uma correla\c{c}\~ao positiva com quest\~oes dif\'{i}ceis;

\end{itemize}
\end{block}

\end{frame}

\begin{frame}
\frametitle{Resultados e Discuss\~oes}
\begin{table}[]\footnotesize
\centering
\caption{Resultado​ ​da​ ​correla\c{c}\~ao​ ​de​ ​Pearson​ ​para​ ​cada​ ​emo\c{c}\~ao​ ​detectada
e​ ​a​ ​entropia​ ​contra​ ​os​ ​atributos​ ​das​ ​quest\~oes}
\label{my-label}
\begin{tabular}{|c|c|c|}
\hline
                      & \textbf{Nível de Dificuldade} & \textbf{Proporção de Acertos} \\ \hline
Tristeza	     & -0.33                & 0.27                          \\ \hline
\small \textbf{Neutralidade} & \small \textbf{0.36}                 & -0.48                \\ \hline
Desprezo     		& -0.15                         & 0.30                 \\ \hline
Desgosto              & -0.13                         & 0.07                          \\ \hline
Raiva                 & -0.14                         & -0.08                         \\ \hline
Surpresa              & 0.07                          & 0.24                          \\ \hline
Medo                  & -0.06                         & 0.14                          \\ \hline
Felicidade   		& -0.14                         & 0.31                 \\ \hline
Entropia     		& -0.12                         & 0.36                 \\ \hline
\end{tabular}
\end{table}
\end{frame}



\begin{frame}
\frametitle{Resultados e Discuss\~oes}
\begin{block}{Neutralidade}
\begin{itemize}
\item A emo\c{c}\~ao mais frequente;
\item Ind\'{i}cio da neutralidade assemelhar com estado de concentra\c{c}\~ao;
\item Neutralidade possui uma correla\c{c}\~ao positiva com quest\~oes dif\'{i}ceis;
\pause
\item Importante identificar os casos diferentes de neutralidade por serem menos frequentes 
e indicar alguma oportunidade de um tutor inteligente atuar.
\end{itemize}
\end{block}

\end{frame}


%FIM RESULTADOS E DISCUSSAO






\section{Trabalhos Futuros}
\begin{frame}
\frametitle{Trabalhos Futuros}
\pause
\begin{itemize}
\item Investigar a mudan\c{c}a temporal dos estados emocionais dos alunos diante de est\'{i}mulos produzidos por diferentes objetos de aprendizagem;
\pause
\item A implementa\c{c}\~ao de uma heur\'istica que considera o estado emocional para fornecer ajuda sob demanda;
\pause
\item Investigar alternativas para amenizar erros de classifica\c{c}\~ao e melhorar acur\'acia do \textit{framework}.
\end{itemize}

\end{frame}

\begin{frame}{Referencias}
\frametitle{Refer\^encias}
    \tiny{\bibliographystyle{apa} }
    \bibliography{base1}
\end{frame}


\section{Agradecimentos}
 \begin{frame}{Agradecimentos}
\begin{itemize}
 \item Esta pesquisa foi parcialmente financiada por intermédio de um projeto de P\&D com a
Samsung Eletr\^{o}nica da Amaz\^{o}nia Ltda; 
\item Os autores Anderson Cruz e Tiago Thompsen
Primo s\~ao bolsistas CAPES e agradecem pelo​ ​apoio​ ​financeiro ​e​ ​concess\~ao​ ​de​ ​bolsa. 
\end{itemize}

\end{frame}

  \begin{frame}{Agradecimentos}
  \begin{center}
 
  \textcolor{VerdeUFAM}{\Large \textbf{Obrigado pela sua aten\c{c}\~ao!}} \\
  \vspace*{20px}
  \textit{\textbf{Anderson Cruz}} \\
  \textit{\textbf{aac@icomp.ufam.edu.br}}
 
  \end{center}
  \end{frame}

 
 \begin{frame}
\frametitle{Slides Adicionais - Tabela 1}
\begin{figure}
\centering
\includegraphics[scale=0.45]{figuras/artigosbie.png}
\end{figure}
\end{frame}

\begin{frame}
\frametitle{Slides Adicionais - Tabela 2}
\begin{figure}
\centering
\includegraphics[scale=0.45]{figuras/artigoSBIE2.png}
\end{figure}

\end{frame}


\begin{frame}
\frametitle{Slides Adicionais - Q1}
\begin{figure}
\centering
\includegraphics[scale=0.55]{figuras/q1.png}
\end{figure}
\end{frame}

\begin{frame}
\frametitle{Slides Adicionais - Q2}
\begin{figure}
\centering
\includegraphics[scale=0.5]{figuras/q2.png}
\end{figure}
\end{frame}

\begin{frame}
\frametitle{Slides Adicionais - Q3}
\begin{figure}
\centering
\includegraphics[scale=0.40]{figuras/q3.png}
\end{figure}
\end{frame}

\begin{frame}
\frametitle{Slides Adicionais - Q4}
\begin{figure}
\centering
\includegraphics[scale=0.55]{figuras/q4.png}
\end{figure}
\end{frame}

\begin{frame}
\frametitle{Slides Adicionais - Q5}
\begin{figure}
\centering
\includegraphics[scale=0.45]{figuras/q5.png}
\end{figure}
\end{frame}

\begin{frame}
\frametitle{Slides Adicionais - Q6}
\begin{figure}
\centering
\includegraphics[scale=0.55]{figuras/q6.png}
\end{figure}
\end{frame}

\begin{frame}
\frametitle{Slides Adicionais - Q7}
\begin{figure}
\centering
\includegraphics[scale=0.40]{figuras/q7.png}
\end{figure}
\end{frame}

\begin{frame}
\frametitle{Slides Adicionais - Q8}
\begin{figure}
\centering
\includegraphics[scale=0.45]{figuras/q8.png}
\end{figure}
\end{frame}

\begin{frame}
\frametitle{Slides Adicionais - Q9}
\begin{figure}
\centering
\includegraphics[scale=0.45]{figuras/q9.png}
\end{figure}
\end{frame}

\begin{frame}
\frametitle{Slides Adicionais - Q11}
\begin{figure}
\centering
\includegraphics[scale=0.45]{figuras/q11.png}
\end{figure}
\end{frame}
 
\end{document}
