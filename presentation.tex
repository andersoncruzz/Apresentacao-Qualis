\documentclass{beamer}

\usepackage{bibentry}


\usepackage{natbib}
\setcitestyle{square, comma, numbers,sort&compress, super}
\usepackage{apalike}


\usepackage{color}				% Controle das cores
\usepackage{amsmath}

\usepackage{breqn}
\usepackage{siunitx}
\usepackage{tikz}
\usetikzlibrary{automata,petri,calc,positioning,shapes.geometric,arrows,shapes,backgrounds}


\usepackage[ruled]{algorithm}
\usepackage{algorithmic}
\usepackage{graphicx}

\usepackage[portuges]{babel}    
\usepackage[utf8]{inputenc}

\usepackage{tikz}
\usetikzlibrary{arrows,automata}
\usepackage{subfig}
\usepackage{pdflscape}
\usepackage{pgfgantt}


\usepackage{pdfpages}
\usepackage{siunitx}
\usepackage{multirow}
\usepackage{color}

\usepackage{ragged2e}


\newcommand*\rot{\rotatebox{90}}
\bibstyle{apa}

\pgfdeclarelayer{background}
\pgfsetlayers{background,main}

\lecture{Uma Abordagem para Reconhecimento de Emoção por Expressão Facial baseada em Redes Neurais de Convolução}{lecture-text}

%\date{October $31^{th}$, 2017}
\date{Manaus, 30 de Julho de 2018}

% Copyright 2007 by Till Tantau
%
% This file may be distributed and/or modified
%
% 1. under the LaTeX Project Public License and/or
% 2. under the GNU Public License.
%
% See the file doc/licenses/LICENSE for more details.


% Common packages

\usepackage{times}
\mode<article>
{
  \usepackage{times}
  \usepackage{mathptmx}
  \usepackage[left=1.5cm,right=6cm,top=1.5cm,bottom=3cm]{geometry}
}

\usepackage{hyperref}
\usepackage[T1]{fontenc}
\usepackage{tikz}
\usepackage{colortbl}
\usepackage{yfonts}
\usepackage{colortbl}
%\usepackage{translator} % comment this, if not available


% Common settings for all lectures in this course

\def\lecturename{XXVIII Simp\'{o}sio Brasileiro de Inform\'{a}tica na Educa\c{c}\~{a}o}

\title{\insertlecture}

\author{Anderson Cruz, Juan Colonna, Gabriel Leit\~{a}o, \\ Edson Silva, Raimundo Barreto, Tiago Primo} 

\institute
{

\begin{tiny}
 Instituto de Computa\c{c}\~ao (IComp), Centro de Desenvolvimento Tecnol\'{o}gico (CDTec)\\
 %Programa de P\'{o}s-Gradua\c{c}\~{a}o em Inform\'{a}tica, Programa de P\'{o}s-Gradua\c{c}\~{a}o em Computa\c{c}\~ao \\
 Universidade Federal do Amazonas (UFAM), Universidade Federal de Pelotas (UFPEL) \\
 Email: <\{aac, juancolonna, gabriel.leitao, rbarreto\}@icomp.ufam.edu.br, \\ edsonaraujo@ufam.edu.br, tiagoprimo@gmail.com > \\
\end{tiny}
}

\subject{\lecturename}




% Beamer version theme settings
\definecolor{VerdeUFAM}{rgb}{0.06,0.41,0.20}
\useoutertheme[height=0pt,width=2cm,left]{sidebar}
\usecolortheme{rose,sidebartab}
\useinnertheme{circles}
\usefonttheme[only large]{structurebold}

\setbeamercolor{sidebar left}{bg=black!15}
\setbeamercolor{structure}{fg=VerdeUFAM}
\setbeamercolor{author}{parent=structure}

\setbeamerfont{title}{series=\normalfont,size=\LARGE}
\setbeamerfont{title in sidebar}{series=\bfseries}
\setbeamerfont{author in sidebar}{series=\bfseries}
\setbeamerfont*{item}{series=}
\setbeamerfont{frametitle}{size=}
\setbeamerfont{block title}{size=\small}
\setbeamerfont{subtitle}{size=\normalsize,series=\normalfont}

\setbeamertemplate{navigation symbols}{}
\setbeamertemplate{bibliography item}[book]
\setbeamertemplate{sidebar left}
{
  \hbox to2cm{\hss\insertlogo\hss}
%  \vskip1.25em%
%   {\usebeamerfont{title in sidebar}%
% %    \vskip1.0em
%     \hskip1pt
%     \usebeamercolor[fg]{title in sidebar}%
%     \insertshorttitle[width=2cm,left,respectlinebreaks]\par%
%     \vskip1.25em%
%   }%
  \insertverticalnavigation{2cm}%
%  \vfill
  \hbox to 2cm{\hfill\usebeamerfont{subsection in
      sidebar}\strut\usebeamercolor[fg]{subsection in
      sidebar}}
  \vskip1pt%
  {%
    \hskip3pt%
    \usebeamercolor[fg]{author in sidebar}%
    \usebeamerfont{author in sidebar}%
    \insertshortauthor[width=2cm,center,respectlinebreaks]\par%
     \vskip4pt%
  }%
  \hbox to 2cm{
  \usebeamercolor[fg]{author in sidebar}%
    \usebeamerfont{author in sidebar}%
\insertframenumber\hskip5pt}%
}%

\setbeamertemplate{title page}
{
  \vbox{}
  \vskip1em
  {\huge \par}
  {\usebeamercolor[fg]{title}\usebeamerfont{title}\inserttitle\par}%
  \ifx\insertsubtitle\@empty%
  \else%
    \vskip0.25em%
    {\usebeamerfont{subtitle}\usebeamercolor[fg]{subtitle}\insertsubtitle\par}%
  \fi%     
  \vskip1em\par
  \emph{\lecturename}\par On: \insertdate\par
  \vskip0pt plus1filll
  \leftskip=0pt plus1fill\insertauthor\par
  \insertinstitute\vskip1em
}

\logo{\includegraphics[width=2cm]{figuras/brasaocor.jpg}}

% Typesetting Listings

\usepackage{listings}
\lstset{language=Java}

\alt<presentation>
{\lstset{%
  basicstyle=\footnotesize\ttfamily,
  commentstyle=\slshape\color{green!50!black},
  keywordstyle=\bfseries\color{blue!50!black},
  identifierstyle=\color{blue},
  stringstyle=\color{orange},
  escapechar=\#,
  emphstyle=\color{red}}
}
{
  \lstset{%
    basicstyle=\ttfamily,
    keywordstyle=\bfseries,
    commentstyle=\itshape,
    escapechar=\#,
    emphstyle=\bfseries\color{red}
  }
}



% Common theorem-like environments

\theoremstyle{definition}
\newtheorem{exercise}[theorem]{\translate{Exercise}}




% New useful definitions:

\newbox\mytempbox
\newdimen\mytempdimen

\newcommand\includegraphicscopyright[3][]{%
  \leavevmode\vbox{\vskip3pt\raggedright\setbox\mytempbox=\hbox{\includegraphics[#1]{#2}}%
    \mytempdimen=\wd\mytempbox\box\mytempbox\par\vskip1pt%
    \fontsize{3}{3.5}\selectfont{\color{black!25}{\vbox{\hsize=\mytempdimen#3}}}\vskip3pt%
}}

\newenvironment{colortabular}[1]{\medskip\rowcolors[]{1}{blue!20}{blue!10}\tabular{#1}\rowcolor{blue!40}}{\endtabular\medskip}

\def\equad{\leavevmode\hbox{}\quad}

\newenvironment{greencolortabular}[1]
{\medskip\rowcolors[]{1}{green!50!black!20}{green!50!black!10}%
  \tabular{#1}\rowcolor{green!50!black!40}}%
{\endtabular\medskip}




\setbeamertemplate{bibliography item}{\insertbiblabel}
\setbeamercovered{transparent}

\begin{document}

\begin{frame}
 \maketitle
\end{frame}

\begin{frame}{Agenda}
  \tableofcontents
\end{frame}

\section{Introdução}

\begin{frame}
\frametitle{Contexto}
\begin{figure}
\centering
\includegraphics[scale=0.39]{figuras/contexto_1.png}
%\caption{Abordagem Proposta}
\label{fig:problema1}
\end{figure}
\end{frame}

\begin{frame}
\frametitle{Contexto}
\begin{figure}
\centering
\includegraphics[scale=0.39]{figuras/facesbasicas.png}
%\caption{Abordagem Proposta}
\label{fig:problema1}
\end{figure}
\end{frame} 



\begin{frame}
\frametitle{Contexto}
\begin{figure}
\centering
\includegraphics[scale=0.39]{figuras/contexto_21.png}
%\caption{Abordagem Proposta}
\label{fig:problema1}
\end{figure}
\end{frame}


%\begin{frame}
%\frametitle{Contexto}
%\begin{figure}
%\centering
%\includegraphics[scale=0.39]{figuras/contexto_3.png}
%%%%\caption{Abordagem Proposta}
%\label{fig:problema1}
%\end{figure}
%\end{frame}

\begin{frame}
\frametitle{Motivação}
\begin{figure}
\centering
\includegraphics[scale=0.39]{figuras/contexto_4.png}
%\caption{Abordagem Proposta}
\label{fig:problema1}
\end{figure}
\end{frame}


\begin{frame}
\frametitle{Problema}

\begin{figure}
\centering
\includegraphics[scale=0.39]{figuras/problema_slide_1.png}
%\caption{Abordagem Proposta}
\label{fig:problema1}
\end{figure}

\end{frame}



\begin{frame}
\frametitle{Problema}

\begin{figure}
\centering
\includegraphics[scale=0.22]{figuras/problema_slide_1.png}
%\caption{Abordagem Proposta}
\label{fig:problema1}
\end{figure}

\begin{figure}
\centering
\includegraphics[scale=0.39]{figuras/problema_slide_2.png}
%\caption{Abordagem Proposta}
\label{fig:arquitetura1}
\end{figure}

%%%IMAGEM DO PROBLEMA MUITO DIFICIL ESSA DEFINIÇÃO


\end{frame}


\begin{frame}
\frametitle{Problema}

\begin{figure}
\centering
\includegraphics[scale=0.22]{figuras/problema_slide_1.png}
%\caption{Abordagem Proposta}
\label{fig:problema1}
\end{figure}

\begin{figure}
\centering
\includegraphics[scale=0.39]{figuras/problema_slide_4.png}
%\caption{Abordagem Proposta}
\label{fig:arquitetura1}
\end{figure}
\end{frame}

\begin{frame}
\frametitle{Problema}
\begin{figure}
\centering
\includegraphics[scale=0.39]{figuras/contexto_5.png}
%\caption{Abordagem Proposta}
\label{fig:problema1}
\end{figure}
\end{frame}


%\begin{frame}
%\frametitle{Problema}
%Problemas clássicos:
%\begin{itemize}
%\pause
% \item Ausência de iluminação no ambiente;
% \pause
% \item Rotação do objeto principal, neste caso a face;
% \pause
% \item Escala do objeto principal (face).
%\end{itemize}
%\end{frame}

\begin{frame}
\frametitle{Problema}
\pause
\textit{Como aprimorar os métodos de reconhecimento de emoções por meio da expressão facial a fim de permitir a classificação independente das características do ambiente e de indivíduos para o alcance de maior generalização?} 


\end{frame}


\begin{frame}
\frametitle{Objetivos}
\begin{block}{Objetivo Geral:}
\begin{itemize}
\pause
\item Propor um método para reconhecer emoção humana por expressão facial para classificar emoções básicas em múltiplas faces de uma imagem e comparar a eficácia em cenários de uso real
\end{itemize}
\end{block}


\frametitle{Objetivos}
\begin{block}{Objetivos Específicos:}
\begin{itemize}
\pause
 \item Propor técnicas de eliminação de ruídos e detecção com recorte das diversas faces de uma imagem
 \pause
 \item Classificar cada face detectada separadamente estimando a probabilidade para cada emoção básica
 \pause
 \item Avaliar experimentalmente a solução proposta visando a comparação da eficácia
\end{itemize}
\end{block}
\end{frame}



\section{Abordagem Proposta}
%\begin{frame}
%\frametitle{Abordagem Proposta}
%\pause
%\begin{block}{Monitoramento}
%aqui;
%\end{block}
%\pause
%\begin{block}{Pré-Processamento}
%aqui;
%\end{block}
%\pause
%\begin{block}{Rede Neural de Convolução}
%aqui;
%\end{block}
%\begin{block}{alguma coisa}
%aqui;
%\end{block}

%\end{frame}


\begin{frame}
\frametitle{Abordagem Proposta - Monitoramento }
\begin{figure}
\centering
\includegraphics[scale=0.37]{figuras/monitoramento_1.png}
%\caption{Abordagem Proposta}
\label{fig:arquitetura1}
\end{figure}
\end{frame}


\begin{frame}
\frametitle{Abordagem Proposta - Monitoramento }
\begin{figure}
\centering
\includegraphics[scale=0.37]{figuras/monitoramento_2.png}
%\caption{Abordagem Proposta}
\label{fig:arquitetura1}
\end{figure}
\end{frame}

\begin{frame}
\frametitle{Abordagem Proposta - Monitoramento }
\begin{figure}
\centering
\includegraphics[scale=0.37]{figuras/monitoramento_3.png}
%\caption{Abordagem Proposta}
\label{fig:arquitetura1}
\end{figure}
\end{frame}



\begin{frame}
\frametitle{Abordagem Proposta - Detecção de Face e Recorte}
\begin{figure}
\centering
\includegraphics[scale=0.37]{figuras/abordagem_4.png}
%\caption{Abordagem Proposta}
\label{fig:arquitetura2}
\end{figure}
\end{frame}

\begin{frame}
\frametitle{Abordagem Proposta - Pré-Processamento}
\begin{figure}
\centering
\includegraphics[scale=0.34]{figuras/abordagem_5.png}
%\caption{Abordagem Proposta}
\label{fig:arquitetura3}
\end{figure}
\end{frame}

\begin{frame}
\frametitle{Abordagem Proposta - Pré-Processamento}
\begin{figure}
\centering
\includegraphics[scale=0.26]{figuras/face_alinhada.png}
\caption{Alinhamento de Face}
\label{fig:face_alinhada}
\end{figure}
\end{frame}



\begin{frame}
\frametitle{Abordagem Proposta - Rede Neural de Convolução}
\begin{figure}
\centering
\includegraphics[scale=0.33]{figuras/abordagem_6.png}
%\caption{Abordagem Proposta}
\label{fig:arquitetura3}
\end{figure}
\end{frame}


\begin{frame}
\frametitle{Abordagem Proposta - Rede Neural de Convolução - Treinamento}
\begin{figure}
\centering
\includegraphics[scale=0.33]{figuras/augmentation.png}
\caption{Aumentação de Dados}
\label{fig:augmentation}
\end{figure}
\end{frame}


\begin{frame}
\frametitle{Abordagem Proposta - Soluções}
\begin{figure}
\centering
\includegraphics[scale=0.35]{figuras/abordagem_7.png}
%\caption{Abordagem Proposta}
\label{fig:arquitetura3}
\end{figure}
\end{frame}



 

\section{Experimento}

\begin{frame}
 \frametitle{Experimento}
\begin{block}{Objetivos}
\begin{itemize}
\pause
\item Implementar parcialmente a abordagem proposta
\pause
\item Realizar um estudo comparativo entre as arquiteturas AlexNet, Inception-V3 e ResNet
\pause
\item Avaliar os modelos usando as métricas de precisão, revocação, f1-score e a acurácia
\pause
\item Analisar os resultados de uma base de dados oriunda da natureza e outra do laboratório

\end{itemize}
\end{block} 
\end{frame}


\begin{frame}
 \frametitle{Experimento}
\begin{block}{Materiais}
\begin{itemize}
\pause
\item \textit{Framework Tensorflow} e \textit{TFLearn}
\pause
\item \textit{OpenCV 3.0}
\pause
\item \textit{GPU NVIDIA GEFORCE 930}, \textit{Intel Core-i7} e \textit{16 GB de RAM DDR4}

\end{itemize}
\end{block} 
\end{frame}



\begin{frame}
\frametitle{Experimento}
\begin{table}
\tiny
\centering
\caption{As bases de dados foram concatenadas e divididas em três bases: treino, teste e validação. Na seguinte porcentagem: 50\% para treino e 25\% para teste e 25\% validação }
\label{table:basesdivisao}
\begin{tabular}{lcccc}
\hline
\textbf{Base de Dados} & \multicolumn{1}{l}{\textbf{B. de Treino}} & \multicolumn{1}{l}{\textbf{B. de Teste}} & \multicolumn{1}{l}{\textbf{B. de Validação}} & \multicolumn{1}{l}{\textbf{Total de Imagens}} \\ \hline
RAFD                   & 2408                                        & 1206                                       & 1205                                           & 4819                                          \\
CIFE-TRAIN             & 4086                                        & 2042                                       & 2042                                           & 8170                                          \\
CIFE-TEST              & 1759                                        & 879                                        & 878                                            & 3516                                          \\
\textbf{CK}                     & \textbf{1509}                                        & \textbf{754}                                        & \textbf{755}                                            & \textbf{3018}                                          \\
KDEF                   & 1466                                        & 735                                        & 733                                            & 2934                                          \\
JAFFE                  & 105                                         & 53                                         & 55                                             & 213                                           \\
NOVAEMOTIONS           & 16840                                       & 8418                                       & 8417                                           & 33675                                         \\
\textbf{FER}                    & \textbf{11782}                                       & \textbf{5892}                                       & \textbf{5891}                                           & \textbf{23565}                                         \\
Total de Imagens       & 39955                                       & 19979                                      & 19976                                          & 79910                                         \\ \hline
\end{tabular}
\end{table}
\end{frame}


\begin{frame}
\frametitle{Experimento}
\begin{table}[]
\tiny
\centering
\caption{Distribuição das classes (emoções) nas bases de treino, teste e validação. As classes também foram divididas em: 50\% para treino e 25\% para teste e 25\% validação}
\label{table:distclasse}
\begin{tabular}{lcccc}
\hline
\textbf{Classe}  & \multicolumn{1}{l}{\textbf{B. de Treino}} & \multicolumn{1}{l}{\textbf{B. de Teste}} & \multicolumn{1}{l}{\textbf{B. de Validação}} & \multicolumn{1}{l}{\textbf{Total de Imagens}} \\ \hline
Raiva            & 3299                                        & 1650                                       & 1650                                           & 6599                                          \\
Desgosto         & 2453                                        & 1226                                       & 1226                                           & 4905                                          \\
Medo             & 2821                                        & 1411                                       & 1410                                           & 5642                                          \\
Felicidade       & 13943                                       & 6971                                       & 6971                                           & 27885                                         \\
Tristeza         & 4349                                        & 2175                                       & 2174                                           & 8698                                          \\
Surpesa          & 6311                                        & 3156                                       & 3155                                           & 12622                                         \\
Neutralidade     & 6779                                        & 3390                                       & 3390                                           & 13559                                         \\
Total de Imagens & 39955                                       & 19979                                      & 19976                                          & 79910                                         \\ \hline
\end{tabular}
\end{table}
\end{frame}


\begin{frame}
\frametitle{Experimento}
\begin{figure}
\centering
\includegraphics[scale=0.32]{figuras/loss-val.png}
\caption{Função de perda na base de Teste}
\label{fig:arquitetura4}
\end{figure}
\end{frame}

%\begin{frame}
%\frametitle{Experimento}
%\begin{figure}
%\centering
%\includegraphics[scale=0.32]{figuras/accuracy_val.png}
%\caption{Acurácia na base de Teste}
%\label{fig:arquitetura4}
%\end{figure}
%\end{frame}



\section{Resultados Parciais}

\begin{frame}
 \frametitle{Resultados Parciais}
\begin{table}[]
\tiny
\centering
\caption{Resultados experimentais das redes neurais de convolução avaliando a base de validação geral}
\label{table:resultsexp}
\begin{tabular}{llcccc}
\hline
\textbf{Arquitetura}                & \textbf{Emoção}       & \textbf{Precisão} & \textbf{Revocação} & \textbf{F1-score} & \textbf{Acurácia}               \\ \hline
\multirow{8}{*}{Alexnet}            & Raiva                 & 0.51              & 0.60               & 0.55              & \multirow{8}{*}{0.712}          \\
                                    & Desgosto              & 0.62              & 0.64               & 0.63              &                                 \\
                                    & Medo                  & 0.47              & 0.41               & 0.44              &                                 \\
                                    & Felicidade            & 0.84              & 0.89               & 0.86              &                                 \\
                                    & Tristeza              & 0.64              & 0.50               & 0.56              &                                 \\
                                    & Surpresa              & 0.84              & 0.77               & 0.80              &                                 \\
                                    & Neutralidade          & 0.62              & 0.64               & 0.63              &                                 \\
                                    & Média/Total           & 0.71              & 0.71               & 0.71              &                                 \\ \hline
\multirow{8}{*}{Inception-V3}       & Raiva                 & 0.54              & 0.51               & 0.52              & \multirow{8}{*}{0.701}          \\
                                    & Desgosto              & 0.56              & 0.57               & 0.56              &                                 \\
                                    & Medo                  & 0.47              & 0.42               & 0.44              &                                 \\
                                    & Felicidade            & 0.88              & 0.88               & 0.88              &                                 \\
                                    & Tristeza              & 0.47              & 0.53               & 0.50              &                                 \\
                                    & Surpresa              & 0.85              & 0.79               & 0.82              &                                 \\
                                    & Neutralidade          & 0.59              & 0.62               & 0.61              &                                 \\
                                    & Média/Total           & 0.70              & 0.70               & 0.70              &                                 \\ \hline
\multirow{8}{*}{\textbf{ResNet-34}} & \textbf{Raiva}        & \textbf{0.69}     & \textbf{0.57}      & \textbf{0.62}     & \multirow{8}{*}{\textbf{0.757}} \\
                                    & \textbf{Desgosto}     & \textbf{0.79}     & \textbf{0.66}      & \textbf{0.72}     &                                 \\
                                    & \textbf{Medo}         & \textbf{0.45}     & \textbf{0.50}      & \textbf{0.47}     &                                 \\
                                    & \textbf{Felicidade}   & \textbf{0.90}     & \textbf{0.89}      & \textbf{0.90}     &                                 \\
                                    & \textbf{Tristeza}     & \textbf{0.60}     & \textbf{0.65}      & \textbf{0.63}     &                                 \\
                                    & \textbf{Surpresa}     & \textbf{0.82}     & \textbf{0.86}      & \textbf{0.84}     &                                 \\
                                    & \textbf{Neutralidade} & \textbf{0.67}     & \textbf{0.68}      & \textbf{0.68}     &                                 \\
                                    & \textbf{Média/Total}  & \textbf{0.76}     & \textbf{0.76}      & \textbf{0.76}     &                                 \\ \hline
\end{tabular}
\end{table}
 
\end{frame}

%\begin{frame}
%\frametitle{Resultados Parciais}
% \begin{block}{Discussão}
%\begin{itemize}
%\pause
%\item A experimentação na base de validação geral assemelha-se a uma medição do cenário real;
%\pause
%\item A ResNet obteve os melhores resultados alcançando 75.7\% em acurácia, enquanto a Alexnet 71.2\% e a Inception-V3 70.1\%;
%\pause
%\item A emoção com melhor desempenho foi a felicidade com f1-score em 90\%;
%\pause 
%\item A segunda melhor emoção foi a surpresa com f1-score em 84\%;
%\pause
%\item São duas das três emoções com maiores índices de amostragem;
%\end{itemize}
%\end{block}
%\end{frame}


\begin{frame}
\frametitle{Resultados Parciais}
 \begin{block}{Discussão}
\begin{itemize}
\pause
\item A experimentação na base de validação geral assemelha-se a uma medição do cenário real
\pause
\item A ResNet obteve os melhores resultados alcançando 75.7\% em acurácia, enquanto a Alexnet 71.2\% e a Inception-V3 70.1\%
\end{itemize}
\end{block}
\end{frame}


\begin{frame}
 \frametitle{Resultados Parciais}
\begin{table}[]
\tiny
\centering
\caption{Resultados experimentais das redes neurais de convolução avaliando a base de validação geral}
\label{table:resultsexp}
\begin{tabular}{llcccc}
\hline
\textbf{Arquitetura}                & \textbf{Emoção}       & \textbf{Precisão} & \textbf{Revocação} & \textbf{F1-score} & \textbf{Acurácia}               \\ \hline
\multirow{8}{*}{Alexnet}            & Raiva                 & 0.51              & 0.60               & 0.55              & \multirow{8}{*}{\scriptsize \textbf{0.712}}          \\
                                    & Desgosto              & 0.62              & 0.64               & 0.63              &                                 \\
                                    & Medo                  & 0.47              & 0.41               & 0.44              &                                 \\
                                    & Felicidade            & 0.84              & 0.89               & 0.86              &                                 \\
                                    & Tristeza              & 0.64              & 0.50               & 0.56              &                                 \\
                                    & Surpresa              & 0.84              & 0.77               & 0.80              &                                 \\
                                    & Neutralidade          & 0.62              & 0.64               & 0.63              &                                 \\
                                    & Média/Total           & 0.71              & 0.71               & 0.71              &                                 \\ \hline
\multirow{8}{*}{Inception-V3}       & Raiva                 & 0.54              & 0.51               & 0.52              & \multirow{8}{*}{\scriptsize \textbf{0.701}}          \\
                                    & Desgosto              & 0.56              & 0.57               & 0.56              &                                 \\
                                    & Medo                  & 0.47              & 0.42               & 0.44              &                                 \\
                                    & Felicidade            & 0.88              & 0.88               & 0.88              &                                 \\
                                    & Tristeza              & 0.47              & 0.53               & 0.50              &                                 \\
                                    & Surpresa              & 0.85              & 0.79               & 0.82              &                                 \\
                                    & Neutralidade          & 0.59              & 0.62               & 0.61              &                                 \\
                                    & Média/Total           & 0.70              & 0.70               & 0.70              &                                 \\ \hline
\multirow{8}{*}{ResNet-34} & Raiva        & 0.69     & 0.57      & 0.62     & \multirow{8}{*}{\scriptsize \textbf{0.757}} \\
                                    & Desgosto     & 0.79     & 0.66      & 0.72     &                                 \\
                                    & Medo         & 0.45     & 0.50      & 0.47     &                                 \\
                                    & Felicidade   & 0.90     & 0.89      & 0.90     &                                 \\
                                    & Tristeza     & 0.60     & 0.65      & 0.63     &                                 \\
                                    & Surpresa     & 0.82     & 0.86      & 0.84     &                                 \\
                                    & Neutralidade & 0.67     & 0.68      & 0.68     &                                 \\
                                    & Média/Total  & 0.76     & 0.76      & 0.76     &                                 \\ \hline
\end{tabular}
\end{table} 
\end{frame}


\begin{frame}
 \frametitle{Resultados Parciais}
\begin{table}[]
\tiny
\centering
\caption{(*) Significa que a rede foi treinada (\emph{fine-tuning}) por duas vezes.}
\label{my-label}
\begin{tabular}{|c|c|c|c|c|}
\hline
\textbf{Arquitetura} & \textbf{Trabalho} & \textbf{Base de Treino} & \textbf{Base de Validação} & \textbf{Acurácia} \\ \hline
\multirow{15}{*}{AlexNet} & \multirow{3}{*}{\cite{art1}} & CK+ & CK+ & 99.1\% \\ \cline{3-5} 
 &  & CK+ & JAFFE & 83.11\% \\ \cline{3-5} 
 &  & JAFFE & JAFFE & 87.7\% \\ \cline{2-5} 
 & \multirow{2}{*}{\cite{art2}} & JAFFE & JAFFE & 76.7\% \\ \cline{3-5} 
 &  & CK+ & CK+ & 80.3\% \\ \cline{2-5} 
 & \cite{art4} & FER & FER & 73.73\% \\ \cline{2-5} 
 & \multirow{2}{*}{\cite{art7}} & FER & FER & 76.9\% \\ \cline{3-5} 
 &  & CK+ & CK+ & 97.3\% \\ \cline{2-5} 
 & \cite{art9} & CK+ & CK+ & 96.04\% \\ \cline{2-5} 
 & \multirow{2}{*}{\cite{art11}} & CK+ & CK+ & 98.7\% \\ \cline{3-5} 
 &  & MMI & MMI & 98.6\% \\ \cline{2-5} 
 & \cite{art13}* & FER/EmotiW & FER/EmotiW & 55.6\% \\ \cline{2-5} 
 & \cite{art14} & CK+/FER & CK+/FER & 86.54\% \\ \cline{2-5} 
 & \multirow{2}{*}{\cite{art15}} & CIFE & CIFE & 81.5\% \\ \cline{3-5} 
 &  & CK+ & CK+ & 83\% \\ \hline
\multirow{2}{*}{VGG} & \cite{art8} & FER+ & FER+ & 84.9\% \\ \cline{2-5} 
 & \cite{art13}* & FER/EmotiW & FER/EmotiW & 52.1\% \\ \hline
GoogLeNet & \cite{art10} & FER/SFEW2.0 & FER/SFEW2.0 & 71.3\% \\ \hline
\multirow{10}{*}{Ensemble} & \multirow{4}{*}{\cite{art3}} & FER & FER-Private & 69.96\% \\ \cline{3-5} 
 &  & FER & CK+ & 76.05\% \\ \cline{3-5} 
 &  & FER & JAFFE & 50.70\% \\ \cline{3-5} 
 &  & FER & EmotiW & 34.09\% \\ \cline{2-5} 
 & \cite{art5} & FER & FER & 65.03\% \\ \cline{2-5} 
 & \multirow{5}{*}{\cite{art6}} & \multirow{5}{*}{FER/SFEW} & FER-Test & 66.67\% \\ \cline{4-5} 
 &  &  & SFEW & 64.84\% \\ \cline{4-5} 
 &  &  & CK+ & 65.54\% \\ \cline{4-5} 
 &  &  & KDEF & 50.66\% \\ \cline{4-5} 
 &  &  & JAFFE & 49.17\% \\ \hline
\end{tabular}
\end{table}
\end{frame}



\begin{frame}
\frametitle{Resultados Parciais}
 \begin{block}{Discussão}
\begin{itemize}
\item A experimentação na base de validação geral assemelha-se a uma medição do cenário real
\item A ResNet obteve os melhores resultados alcançando 75.7\% em acurácia, enquanto a Alexnet 71.2\% e a Inception-V3 70.1\%
\pause
\item A emoção com melhor desempenho foi a felicidade com f1-score em 90\%
\end{itemize}
\end{block}
\end{frame}


\begin{frame}
 \frametitle{Resultados Parciais}
\begin{table}[]
\tiny
\centering
\caption{Resultados experimentais das redes neurais de convolução avaliando a base de validação geral}
\label{table:resultsexp}
\begin{tabular}{llcccc}
\hline
\textbf{Arquitetura}                & \textbf{Emoção}       & \textbf{Precisão} & \textbf{Revocação} & \textbf{F1-score} & \textbf{Acurácia}               \\ \hline
\multirow{8}{*}{Alexnet}            & Raiva                 & 0.51              & 0.60               & 0.55              & \multirow{8}{*}{0.712}          \\
                                    & Desgosto              & 0.62              & 0.64               & 0.63              &                                 \\
                                    & Medo                  & 0.47              & 0.41               & 0.44              &                                 \\
                                    & \scriptsize \textbf{Felicidade}            & \scriptsize \textbf{0.84}              & \scriptsize \textbf{0.89}               & \scriptsize \textbf{0.86}              &                                 \\
                                    & Tristeza              & 0.64              & 0.50               & 0.56              &                                 \\
                                    & Surpresa              & 0.84              & 0.77               & 0.80              &                                 \\
                                    & Neutralidade          & 0.62              & 0.64               & 0.63              &                                 \\
                                    & Média/Total           & 0.71              & 0.71               & 0.71              &                                 \\ \hline
\multirow{8}{*}{Inception-V3}       & Raiva                 & 0.54              & 0.51               & 0.52              & \multirow{8}{*}{0.701}          \\
                                    & Desgosto              & 0.56              & 0.57               & 0.56              &                                 \\
                                    & Medo                  & 0.47              & 0.42               & 0.44              &                                 \\
                                    & \scriptsize \textbf{Felicidade}            & \scriptsize \textbf{0.88}              & \scriptsize \textbf{0.88}               & \scriptsize \textbf{0.88}              &                                 \\
                                    & Tristeza              & 0.47              & 0.53               & 0.50              &                                 \\
                                    & Surpresa              & 0.85              & 0.79               & 0.82              &                                 \\
                                    & Neutralidade          & 0.59              & 0.62               & 0.61              &                                 \\
                                    & Média/Total           & 0.70              & 0.70               & 0.70              &                                 \\ \hline
\multirow{8}{*}{ResNet-34} & Raiva        & 0.69     & 0.57      & 0.62     & \multirow{8}{*}{0.757} \\
                                    & Desgosto     & 0.79     & 0.66      & 0.72     &                                 \\
                                    & Medo         & 0.45     & 0.50      & 0.47     &                                 \\
                                    & \scriptsize \textbf{Felicidade}   & \scriptsize \textbf{0.90}     & \scriptsize \textbf{0.89}      & \scriptsize \textbf{0.90}     &                                 \\
                                    & Tristeza     & 0.60     & 0.65      & 0.63     &                                 \\
                                    & Surpresa     & 0.82     & 0.86      & 0.84     &                                 \\
                                    & Neutralidade & 0.67     & 0.68      & 0.68     &                                 \\
                                    & Média/Total  & 0.76     & 0.76      & 0.76     &                                 \\ \hline
\end{tabular}
\end{table} 
\end{frame}





\begin{frame}
\frametitle{Resultados Parciais}
 \begin{block}{Discussão}
\begin{itemize}
\item A experimentação na base de validação geral assemelha-se a uma medição do cenário real
\item A ResNet obteve os melhores resultados alcançando 75.7\% em acurácia, enquanto a Alexnet 71.2\% e a Inception-V3 70.1\%
\item A emoção com melhor desempenho foi a felicidade com f1-score em 90\%
\pause 
\item A segunda melhor emoção foi a surpresa com f1-score em 84\%
\end{itemize}
\end{block}
\end{frame}


\begin{frame}
 \frametitle{Resultados Parciais}
\begin{table}[]
\tiny
\centering
\caption{Resultados experimentais das redes neurais de convolução avaliando a base de validação geral}
\label{table:resultsexp}
\begin{tabular}{llcccc}
\hline
\textbf{Arquitetura}                & \textbf{Emoção}       & \textbf{Precisão} & \textbf{Revocação} & \textbf{F1-score} & \textbf{Acurácia}               \\ \hline
\multirow{8}{*}{Alexnet}            & Raiva                 & 0.51              & 0.60               & 0.55              & \multirow{8}{*}{0.712}          \\
                                    & Desgosto              & 0.62              & 0.64               & 0.63              &                                 \\
                                    & Medo                  & 0.47              & 0.41               & 0.44              &                                 \\
                                    & Felicidade            & 0.84              & 0.89               & 0.86              &                                 \\
                                    & Tristeza              & 0.64              & 0.50               & 0.56              &                                 \\
                                    & \scriptsize \textbf{Surpresa}              & \scriptsize \textbf{0.84}              & \scriptsize \textbf{0.77}               & \scriptsize \textbf{0.80}              &                                 \\
                                    & Neutralidade          & 0.62              & 0.64               & 0.63              &                                 \\
                                    & Média/Total           & 0.71              & 0.71               & 0.71              &                                 \\ \hline
\multirow{8}{*}{Inception-V3}       & Raiva                 & 0.54              & 0.51               & 0.52              & \multirow{8}{*}{0.701}          \\
                                    & Desgosto              & 0.56              & 0.57               & 0.56              &                                 \\
                                    & Medo                  & 0.47              & 0.42               & 0.44              &                                 \\
                                    & Felicidade            & 0.88              & 0.88               & 0.88              &                                 \\
                                    & Tristeza              & 0.47              & 0.53               & 0.50              &                                 \\
                                    & \scriptsize \textbf{Surpresa}              & \scriptsize \textbf{0.85}              & \scriptsize \textbf{0.79}               & \scriptsize \textbf{0.82}              &                                 \\
                                    & Neutralidade          & 0.59              & 0.62               & 0.61              &                                 \\
                                    & Média/Total           & 0.70              & 0.70               & 0.70              &                                 \\ \hline
\multirow{8}{*}{ResNet-34} & Raiva        & 0.69     & 0.57      & 0.62     & \multirow{8}{*}{0.757} \\
                                    & Desgosto     & 0.79     & 0.66      & 0.72     &                                 \\
                                    & Medo         & 0.45     & 0.50      & 0.47     &                                 \\
                                    & Felicidade   & 0.90     & 0.89      & 0.90     &                                 \\
                                    & Tristeza     & 0.60     & 0.65      & 0.63     &                                 \\
                                    & \scriptsize \textbf{Surpresa}     & \scriptsize \textbf{0.82}     & \scriptsize \textbf{0.86}      & \scriptsize \textbf{0.84}     &                                 \\
                                    & Neutralidade & 0.67     & 0.68      & 0.68     &                                 \\
                                    & Média/Total  & 0.76     & 0.76      & 0.76     &                                 \\ \hline
\end{tabular}
\end{table} 
\end{frame}




\begin{frame}
\frametitle{Resultados Parciais}
 \begin{block}{Discussão}
\begin{itemize}
\item A experimentação na base de validação geral assemelha-se a uma medição do cenário real
\item A ResNet obteve os melhores resultados alcançando 75.7\% em acurácia, enquanto a Alexnet 71.2\% e a Inception-V3 70.1\%
\item A emoção com melhor desempenho foi a felicidade com f1-score em 90\%
\item A segunda melhor emoção foi a surpresa com f1-score em 84\%
\pause
\item São duas das três emoções com maiores índices de amostragem
\end{itemize}
\end{block}
\end{frame}


\begin{frame}
\frametitle{Resultados Parciais}
\begin{table}[]
\tiny
\centering
\caption{Distribuição das classes (emoções) nas bases de treino, teste e validação. As classes também foram divididas em: 50\% para treino e 25\% para teste e 25\% validação}
\label{table:distclasse}
\begin{tabular}{lcccc}
\hline
\textbf{Classe}  & \multicolumn{1}{l}{\textbf{B. de Treino}} & \multicolumn{1}{l}{\textbf{B. de Teste}} & \multicolumn{1}{l}{\textbf{B. de Validação}} & \multicolumn{1}{l}{\textbf{Total de Imagens}} \\ \hline
Raiva            & 3299                                        & 1650                                       & 1650                                           & 6599                                          \\
Desgosto         & 2453                                        & 1226                                       & 1226                                           & 4905                                          \\
Medo             & 2821                                        & 1411                                       & 1410                                           & 5642                                          \\
\scriptsize \textbf{Felicidade}       & \scriptsize \textbf{13943}                                       & \scriptsize \textbf{6971}                                       & \scriptsize \textbf{6971}                                           & \scriptsize \textbf{27885}                                         \\
Tristeza         & 4349                                        & 2175                                       & 2174                                           & 8698                                          \\
\scriptsize \textbf{Surpesa}          & \scriptsize \textbf{6311}                                        & \scriptsize \textbf{3156}                                       & \scriptsize \textbf{3155}                                           & \scriptsize \textbf{12622}                                         \\
\scriptsize Neutralidade     & \scriptsize 6779                                        & \scriptsize 3390                                       & \scriptsize 3390                                           & \scriptsize 13559                                         \\
Total de Imagens & 39955                                       & 19979                                      & 19976                                          & 79910                                         \\ \hline
\end{tabular}
\end{table}
\end{frame}



\begin{frame}
\frametitle{Resultados Parciais}
 \begin{block}{Discussão}
\begin{itemize}
\pause
\item Em contrapartida, os piores índices de desempenho foram as emoções medo e raiva

\end{itemize}
\end{block}
\end{frame}



\begin{frame}
 \frametitle{Resultados Parciais}
\begin{table}[]
\tiny
\centering
\caption{Resultados experimentais das redes neurais de convolução avaliando a base de validação geral}
\label{table:resultsexp}
\begin{tabular}{llcccc}
\hline
\textbf{Arquitetura}                & \textbf{Emoção}       & \textbf{Precisão} & \textbf{Revocação} & \textbf{F1-score} & \textbf{Acurácia}               \\ \hline
\multirow{8}{*}{Alexnet}            & \scriptsize \textbf{Raiva}                 & \scriptsize \textbf{0.51}              & \scriptsize \textbf{0.60}               & \scriptsize \textbf{0.55}              & \multirow{8}{*}{0.712}          \\
                                    & Desgosto              & 0.62              & 0.64               & 0.63              &                                 \\
                                    & \scriptsize \textbf{Medo}                  & \scriptsize \textbf{0.47}              & \scriptsize \textbf{0.41}               & \scriptsize \textbf{0.44}              &                                 \\
                                    & Felicidade            & 0.84              & 0.89               & 0.86              &                                 \\
                                    & Tristeza              & 0.64              & 0.50               & 0.56              &                                 \\
                                    & Surpresa              & 0.84              & 0.77               & 0.80              &                                 \\
                                    & Neutralidade          & 0.62              & 0.64               & 0.63              &                                 \\
                                    & Média/Total           & 0.71              & 0.71               & 0.71              &                                 \\ \hline
\multirow{8}{*}{Inception-V3}       & \scriptsize \textbf{Raiva}                 & \scriptsize \textbf{0.54}              & \scriptsize \textbf{0.51}               & \scriptsize \textbf{0.52}              & \multirow{8}{*}{0.701}          \\
                                    & Desgosto              & 0.56              & 0.57               & 0.56              &                                 \\
                                    & \scriptsize \textbf{Medo}                  & \scriptsize \textbf{0.47}              & \scriptsize \textbf{0.42}               & \scriptsize \textbf{0.44}              &                                 \\
                                    & Felicidade            & 0.88              & 0.88               & 0.88              &                                 \\
                                    & Tristeza              & 0.47              & 0.53               & 0.50              &                                 \\
                                    & Surpresa              & 0.85              & 0.79               & 0.82              &                                 \\
                                    & Neutralidade          & 0.59              & 0.62               & 0.61              &                                 \\
                                    & Média/Total           & 0.70              & 0.70               & 0.70              &                                 \\ \hline
\multirow{8}{*}{ResNet-34} & \scriptsize \textbf{Raiva}        & \scriptsize \textbf{0.69}     & \scriptsize \textbf{0.57}      & \scriptsize \textbf{0.62}     & \multirow{8}{*}{0.757} \\
                                    & Desgosto     & 0.79     & 0.66      & 0.72     &                                 \\
                                    & \scriptsize \textbf{Medo}         & \scriptsize \textbf{0.45}     & \scriptsize \textbf{0.50}      & \scriptsize \textbf{0.47}     &                                 \\
                                    & Felicidade   & 0.90     & 0.89      & 0.90     &                                 \\
                                    & Tristeza     & 0.60     & 0.65      & 0.63     &                                 \\
                                    & Surpresa     & 0.82     & 0.86      & 0.84     &                                 \\
                                    & Neutralidade & 0.67     & 0.68      & 0.68     &                                 \\
                                    & Média/Total  & 0.76     & 0.76      & 0.76     &                                 \\ \hline
\end{tabular}
\end{table} 
\end{frame}


\begin{frame}
\frametitle{Resultados Parciais}
 \begin{block}{Discussão}
\begin{itemize}
\item Em contrapartida, os piores índices de desempenho foram as emoções medo e raiva
\pause
\item Coincidentemente duas das três emoções com menores índices de amostragem
\end{itemize}
\end{block}
\end{frame}



\begin{frame}
\frametitle{Experimento}
\begin{table}[]
\tiny
\centering
\caption{Distribuição das classes (emoções) nas bases de treino, teste e validação. As classes também foram divididas em: 50\% para treino e 25\% para teste e 25\% validação}
\label{table:distclasse}
\begin{tabular}{lcccc}
\hline
\textbf{Classe}  & \multicolumn{1}{l}{\textbf{B. de Treino}} & \multicolumn{1}{l}{\textbf{B. de Teste}} & \multicolumn{1}{l}{\textbf{B. de Validação}} & \multicolumn{1}{l}{\textbf{Total de Imagens}} \\ \hline
\scriptsize \textbf{Raiva}            & \scriptsize \textbf{3299}                                        & \scriptsize \textbf{1650}                                       & \scriptsize \textbf{1650}                                           & \scriptsize \textbf{6599}                                          \\
\scriptsize Desgosto         & \scriptsize 2453                                        & \scriptsize 1226                                       & \scriptsize 1226                                           & \scriptsize 4905                                          \\
\scriptsize \textbf{Medo}             & \scriptsize \textbf{2821}                                        & \scriptsize \textbf{1411}                                       & \scriptsize \textbf{1410}                                           & \scriptsize \textbf{5642}                                          \\
Felicidade       & 13943                                       & 6971                                       & 6971                                           & 27885                                         \\
Tristeza         & 4349                                        & 2175                                       & 2174                                           & 8698                                          \\
Surpesa          & 6311                                        & 3156                                       & 3155                                           & 12622                                         \\
Neutralidade     & 6779                                        & 3390                                       & 3390                                           & 13559                                         \\
Total de Imagens & 39955                                       & 19979                                      & 19976                                          & 79910                                         \\ \hline
\end{tabular}
\end{table}
\end{frame}


\begin{frame}
\frametitle{Resultados Parciais}
 \begin{block}{Discussão}
\begin{itemize}
\item Em contrapartida, os piores índices de desempenho foram as emoções medo e raiva
\item Coincidentemente duas das três emoções com menores índices de amostragem
\pause
\item Entretanto, realmente são expressões faciais complicadas. Não havendo um padrão bem definido envolvendo vários músculos faciais
\pause
\item Um sistema com requisito de alta precisão nessas duas emoções necessita de outras fontes de dados
\end{itemize}
\end{block}
\end{frame}


\begin{frame}
\frametitle{Resultados Parciais}
 \begin{block}{Discussão}
\begin{itemize}
\pause
\item A neutralidade foi uma das três emoções que mais teve amostragem. No entanto, alcançou somente 68\% de f1-score
\end{itemize}
\end{block}
\end{frame}



\begin{frame}
 \frametitle{Resultados Parciais}
\begin{table}[]
\tiny
\centering
\caption{Resultados experimentais das redes neurais de convolução avaliando a base de validação geral}
\label{table:resultsexp}
\begin{tabular}{llcccc}
\hline
\textbf{Arquitetura}                & \textbf{Emoção}       & \textbf{Precisão} & \textbf{Revocação} & \textbf{F1-score} & \textbf{Acurácia}               \\ \hline
\multirow{8}{*}{Alexnet}            & Raiva                 & 0.51              & 0.60               & 0.55              & \multirow{8}{*}{0.712}          \\
                                    & Desgosto              & 0.62              & 0.64               & 0.63              &                                 \\
                                    & Medo                  & 0.47              & 0.41               & 0.44              &                                 \\
                                    & Felicidade            & 0.84              & 0.89               & 0.86              &                                 \\
                                    & Tristeza              & 0.64              & 0.50               & 0.56              &                                 \\
                                    & Surpresa              & 0.84              & 0.77               & 0.80              &                                 \\
                                    & \scriptsize \textbf{Neutralidade}          & \scriptsize \textbf{0.62}              & \scriptsize \textbf{0.64}               & \scriptsize \textbf{0.63}              &                                 \\
                                    & Média/Total           & 0.71              & 0.71               & 0.71              &                                 \\ \hline
\multirow{8}{*}{Inception-V3}       & Raiva                 & 0.54              & 0.51               & 0.52              & \multirow{8}{*}{0.701}          \\
                                    & Desgosto              & 0.56              & 0.57               & 0.56              &                                 \\
                                    & Medo                  & 0.47              & 0.42               & 0.44              &                                 \\
                                    & Felicidade            & 0.88              & 0.88               & 0.88              &                                 \\
                                    & Tristeza              & 0.47              & 0.53               & 0.50              &                                 \\
                                    & Surpresa              & 0.85              & 0.79               & 0.82              &                                 \\
                                    & \scriptsize \textbf{Neutralidade}          & \scriptsize \textbf{0.59}              & \scriptsize \textbf{0.62}               & \scriptsize \textbf{0.61}              &                                 \\
                                    & Média/Total           & 0.70              & 0.70               & 0.70              &                                 \\ \hline
\multirow{8}{*}{ResNet-34} & Raiva        & 0.69     & 0.57      & 0.62     & \multirow{8}{*}{0.757} \\
                                    & Desgosto     & 0.79     & 0.66      & 0.72     &                                 \\
                                    & Medo         & 0.45     & 0.50      & 0.47     &                                 \\
                                    & Felicidade   & 0.90     & 0.89      & 0.90     &                                 \\
                                    & Tristeza     & 0.60     & 0.65      & 0.63     &                                 \\
                                    & Surpresa     & 0.82     & 0.86      & 0.84     &                                 \\
                                    & \scriptsize \textbf{Neutralidade} & \scriptsize \textbf{0.67}     & \scriptsize \textbf{0.68}      & \scriptsize \textbf{0.68}     &                                 \\
                                    & Média/Total  & 0.76     & 0.76      & 0.76     &                                 \\ \hline
\end{tabular}
\end{table} 
\end{frame}


\begin{frame}
\frametitle{Resultados Parciais}
 \begin{block}{Discussão}
\begin{itemize}
\item A neutralidade foi uma das três emoções que mais teve amostragem. No entanto, alcançou somente 68\% de f1-score
\pause
\item A rede neural aprendeu que a neutralidade consiste em nenhum movimento muscular facial
\pause
\item Portanto, um leve movimento facial a rede tende a classificar em uma das emoções que não seja neutralidade
\end{itemize}
\end{block}
\end{frame}



\begin{frame}
 \frametitle{Resultados Parciais}
\begin{table}[]
\tiny
\centering
\caption{Resultados experimentais das redes neurais de convolução avaliando a base de validação CK}
\label{table:ck}
\begin{tabular}{llcccc}
\hline
\textbf{Arquitetura}                   & \textbf{Emoção}       & \multicolumn{1}{l}{\textbf{Precisão}} & \multicolumn{1}{l}{\textbf{Revocação}} & \multicolumn{1}{l}{\textbf{F1-score}} & \multicolumn{1}{l}{\textbf{Acurácia}} \\ \hline
\multirow{8}{*}{Alexnet}         & Raiva                 & 0.91                                  & 1                                      & 0.95                                  & \multirow{8}{*}{0.96}                 \\
                                       & Desgosto              & 0.98                                  & 0.97                                   & 0.98                                  &                                       \\
                                       & Medo                  & 0.89                                  & 0.96                                   & 0.92                                  &                                       \\
                                       & Felicidade            & 0.99                                  & 0.99                                   & 0.99                                  &                                       \\
                                       & Tristeza              & 0.98                                  & 0.84                                   & 0.91                                  &                                       \\
                                       & Surpresa              & 1                                     & 0.94                                   & 0.97                                  &                                       \\
                                       & Neutralidade          & 0                                     & 0                                      & 0                                     &                                       \\
                                       & Média/Total           & 0.97                                  & 0.96                                   & 0.96                                  &                                       \\ \hline
\multirow{8}{*}{Inception-V3}     & Raiva                 & 0.93                                  & 0.94                                   & 0.93                                  & \multirow{8}{*}{0.954}                \\
                                       & Desgosto              & 0.96                                  & 0.94                                   & 0.95                                  &                                       \\
                                       & Medo                  & 0.89                                  & 0.96                                   & 0.92                                  &                                       \\
                                       & Felicidade            & 0.99                                  & 0.98                                   & 0.99                                  &                                       \\
                                       & Tristeza              & 0.91                                  & 0.97                                   & 0.94                                  &                                       \\
                                       & Surpresa              & 1                                     & 0.94                                   & 0.97                                  &                                       \\
                                       & Neutralidade          & 0                                     & 0                                      & 0                                     &                                       \\
                                       & Média/Total           & 0.96                                  & 0.95                                   & 0.96                                  &                                       \\ \hline
\multirow{8}{*}{\textbf{ResNet-34}} & \textbf{Raiva}        & \textbf{0.97}                         & \textbf{0.96}                          & \textbf{0.97}                         & \multirow{8}{*}{\textbf{0.969}}       \\
                                       & \textbf{Desgosto}     & \textbf{1}                            & \textbf{0.92}                          & \textbf{0.96}                         &                                       \\
                                       & \textbf{Medo}         & \textbf{0.91}                         & \textbf{0.99}                          & \textbf{0.95}                         &                                       \\
                                       & \textbf{Felicidade}   & \textbf{0.98}                         & \textbf{0.99}                          & \textbf{0.99}                         &                                       \\
                                       & \textbf{Tristeza}     & \textbf{0.94}                         & \textbf{0.96}                          & \textbf{0.95}                         &                                       \\
                                       & \textbf{Surpresa}     & \textbf{0.98}                         & \textbf{0.99}                          & \textbf{0.99}                         &                                       \\
                                       & \textbf{Neutralidade} & \textbf{0}                            & \textbf{0}                             & \textbf{0}                            &                                       \\
                                       & \textbf{Média/Total}  & \textbf{0.97}                         & \textbf{0.97}                          & \textbf{0.97}                         &                                       \\ \hline
\end{tabular}
\end{table} 
 
\end{frame}


\begin{frame}
\frametitle{Resultados Parciais}
 \begin{block}{Discussão}
\begin{itemize}
\item A base CK é de origem laboratorial
\pause
\item A ResNet alcançou 96.9\% de acurácia, enquanto a Alexnet 96\% e a Inception 95.4\%
\pause
\item Os trabalhos de \cite{art1}, \cite{art11} e \cite{art7} obtiveram  99.1\%, 98.7\% e 97.3\%, respectivamente, embora não está claro se é na base de teste ou validação
\pause
\item Não possui amostra da emoção neutralidade
\pause
\item Não houve confusão com a emoção neutralidade
\end{itemize}
\end{block}
\end{frame}


\begin{frame}
 \frametitle{Resultados Parciais}
 \begin{table}[]
 \tiny
\centering
\caption{Resultados experimentais das redes neurais de convolução avaliando a base de validação FER}
\label{table:fer}
\begin{tabular}{llcccc}
\hline
\textbf{Arquitetura}                   & \textbf{Emoção}       & \multicolumn{1}{l}{\textbf{Precisão}} & \multicolumn{1}{l}{\textbf{Revocação}} & \multicolumn{1}{l}{\textbf{F1-score}} & \multicolumn{1}{l}{\textbf{Acurácia}} \\ \hline
\multirow{8}{*}{Alexnet}         & Raiva                 & 0.39                                  & 0.5                                    & 0.44                                  & \multirow{8}{*}{0.543}                \\
                                       & Desgosto              & 0.45                                  & 0.17                                   & 0.25                                  &                                       \\
                                       & Medo                  & 0.37                                  & 0.33                                   & 0.35                                  &                                       \\
                                       & Felicidade            & 0.74                                  & 0.79                                   & 0.76                                  &                                       \\
                                       & Tristeza              & 0.4                                   & 0.29                                   & 0.34                                  &                                       \\
                                       & Surpresa              & 0.72                                  & 0.64                                   & 0.67                                  &                                       \\
                                       & Neutralidade          & 0.49                                  & 0.52                                   & 0.51                                  &                                       \\
                                       & Média/Total           & 0.54                                  & 0.54                                   & 0.54                                  &                                       \\ \hline
\multirow{8}{*}{Inception-V3}     & Raiva                 & 0.43                                  & 0.4                                    & 0.41                                  & \multirow{8}{*}{0.529}                \\
                                       & Desgosto              & 0.13                                  & 0.25                                   & 0.17                                  &                                       \\
                                       & Medo                  & 0.39                                  & 0.36                                   & 0.37                                  &                                       \\
                                       & Felicidade            & 0.81                                  & 0.77                                   & 0.79                                  &                                       \\
                                       & Tristeza              & 0.29                                  & 0.38                                   & 0.32                                  &                                       \\
                                       & Surpresa              & 0.72                                  & 0.64                                   & 0.68                                  &                                       \\
                                       & Neutralidade          & 0.49                                  & 0.46                                   & 0.47                                  &                                       \\
                                       & Média/Total           & 0.55                                  & 0.53                                   & 0.54                                  &                                       \\ \hline
\multirow{8}{*}{\textbf{ResNet-34}} & \textbf{Raiva}        & \textbf{0.61}                         & \textbf{0.42}                          & \textbf{0.5}                          & \multirow{8}{*}{\textbf{0.604}}       \\
                                       & \textbf{Desgosto}     & \textbf{0.69}                         & \textbf{0.28}                          & \textbf{0.39}                         &                                       \\
                                       & \textbf{Medo}         & \textbf{0.35}                         & \textbf{0.47}                          & \textbf{0.4}                          &                                       \\
                                       & \textbf{Felicidade}   & \textbf{0.87}                         & \textbf{0.81}                          & \textbf{0.84}                         &                                       \\
                                       & \textbf{Tristeza}     & \textbf{0.41}                         & \textbf{0.42}                          & \textbf{0.41}                         &                                       \\
                                       & \textbf{Surpresa}     & \textbf{0.71}                         & \textbf{0.76}                          & \textbf{0.73}                         &                                       \\
                                       & \textbf{Neutralidade} & \textbf{0.56}                         & \textbf{0.61}                          & \textbf{0.58}                         &                                       \\
                                       & \textbf{Média/Total}  & \textbf{0.62}                         & \textbf{0.6}                           & \textbf{0.61}                         &                                       \\ \hline
\end{tabular}
\end{table}
\end{frame}


\begin{frame}
\frametitle{Resultados Parciais}
 \begin{block}{Discussão}
\begin{itemize}
\pause
\item A base FER é de origem natural
\pause
\item Analisando a acurácia a ResNet-34 alcançou 60.4\%, enquanto a Alexnet 54.3\% e a Inception-V3 52.9\%
\pause
\item O trabalho de \cite{art7} alcançou 76.9\%, \cite{kim2016fusing} conseguiu 73.73\% e \cite{art5} atingiu 65\%, todos estes trabalhos utilizaram a base FER para treino e teste
\pause
\item A base FER é uma base difícil para ser aprendida e classificada
\pause
\item As faces não tem um padrão de iluminação, variação do fundo do ambiente, distância fixa com câmera, faces com fortes movimentações e emoções com grau de intensidades diferentes
\end{itemize}
\end{block}
\end{frame}


\section{Conclusão}
\begin{frame}
\frametitle{Conclusão}
\begin{block}{Considerações}
\begin{itemize}
\pause 
\item A ResNet foi a melhor arquitetura
\pause
\item Imagens oriundas da natureza não influenciam negativamente para classificação em laboratório
\pause
\item Um método treinado com as imagens laboratoriais não consegue reconhecer emoções no contexto da natureza
\pause
\item Os resultados devem melhorar com a implementação das técnicas como normalização do brilho e contraste, alinhamento da face e aumentação de dados

\end{itemize}
\end{block}

\end{frame}


\begin{frame}
 \frametitle{Cronograma e Trabalhos futuros}

\begin{table}\footnotesize
\tiny
\caption{Cronograma de Atividades}
\label{table:cronog}
\begin{tabular}{l|c|c|c|c|c|c|c|c|}
\cline{2-9}
                                                                           & \multicolumn{5}{c|}{\textbf{2018}}                                       & \multicolumn{3}{c|}{\textbf{2019}}         \\ \hline
\multicolumn{1}{|m{3cm}|}{\textbf{Atividades}}                                  & \textbf{ago} & \textbf{set} & \textbf{out} & \textbf{nov} & \textbf{dez} & \textbf{jan} & \textbf{fev} & \textbf{mar} \\ \hline
\multicolumn{1}{|m{3cm}|}{Desenvolver e avaliar o componente pré-processamento} & x            &              &              &              &              &              &              &              \\
\hline
\multicolumn{1}{|l|}{Analisar sequência de imagens}                        &              & x            &              &              &              &              &              &              \\
\hline
\multicolumn{1}{|m{3cm}|}{Avaliar experimentalmente outros classificadores}     &              &              & x            &              &              &              &              &              \\
\hline
\multicolumn{1}{|l|}{Implementar e avaliar a MobileNet}                    &              &              &              & x            &              &              &              &              \\
\hline
\multicolumn{1}{|l|}{Avaliar em cenários de uso reais}                     &              &              &              &              & x            & x            &              &              \\
\hline
\multicolumn{1}{|l|}{Escrita da dissertação}                                  &              &              &              & x            & x            & x            & x            & x            \\ \hline
\end{tabular}
\end{table} 
 
\end{frame}



\begin{frame}{Referencias}
\frametitle{Referências}
    \tiny{\bibliographystyle{apa} }
    \bibliography{base1}
\end{frame}


\section{Agradecimentos}
\begin{frame}{Agradecimentos}
\begin{itemize}
 \item Ao orientador: prof. Barreto
\item À banca: prof. Elaine e prof. Daniel
\item À minha esposa: Giselle
\item Amigos do grupo de pesquisa
\item À plateia
\end{itemize}

\end{frame}

  \begin{frame}{Agradecimentos}
  \begin{center}
 
  \textcolor{VerdeUFAM}{\Large \textbf{Obrigado pela sua atenção!}} \\
  \vspace*{20px}
  \textit{\textbf{Anderson Cruz}} \\
  \textit{\textbf{aac@icomp.ufam.edu.br}}
 
  \end{center}
  \end{frame}

 
 %\section{Prova de Conceito}

\begin{frame}
\frametitle{Slides Adicionais - Prova de Conceito - Conceitos}
\begin{block}{Entropia}
Uma medida de dispers\~ao que seu resultado implica que quanto maior a entropia, maior \'e a uniformidade da distribui\c{c}\~ao dos dados 
\end{block}
\begin{figure}
\centering
\includegraphics[scale=0.35]{figuras/entropiaSBIE.png}
\end{figure}
\end{frame}

\begin{frame}
\frametitle{Slides Adicionais - Prova de Conceito - Conceitos}
\begin{block}{Correla\c{c}\~ao de Pearson}
Mede o grau de correla\c{c}\~ao (for\c{c}a e dire\c{c}\~ao) entre um par de vari\'aveis aleat\'orias
\end{block}
\begin{figure}
\centering
\includegraphics[scale=0.4]{figuras/pearson.png}
\end{figure}

\end{frame}



%\begin{frame}
% \frametitle{Prova de Conceito}
% \begin{block}{Objetivos}
%\begin{itemize}
%%%\justify
%\pause
%\item Propor um framework para detectar estados emocionais de estudantes baseado em reconhecimento de expressões faciais no contexto das plataformas educacionais
%\pause
%\item Monitorar as emoções dos estudantes durante uma avaliação de múltipla escolha (simulado do ENEM)
%\pause
%\item Correlacionar os estados emocionais detectados e a entropia das emoções com o desempenho durante a avaliação
%\end{itemize}
%\end{block}
%\end{frame}


\begin{frame}
\frametitle{Slides Adicionais - Prova de Conceito - Discussões}
\begin{table}[]\footnotesize
\centering
\caption{Resultado​ ​da​ ​correla\c{c}\~ao​ ​de​ ​Pearson​ ​para​ ​cada​ ​emo\c{c}\~ao​ ​detectada
e​ ​a​ ​entropia​ ​contra​ ​os​ ​atributos​ ​das​ ​quest\~oes}
\label{my-label}
\begin{tabular}{|c|c|c|}
\hline
                      & \textbf{Nível de Dificuldade} & \textbf{Proporção de Acertos} \\ \hline
\textbf{Tristeza}     & \textbf{-0.33}                & 0.27                          \\ \hline
\textbf{Neutralidade} & \textbf{0.36}                 & \textbf{-0.48}                \\ \hline
\textbf{Desprezo}     & -0.15                         & \textbf{0.30}                 \\ \hline
Desgosto              & -0.13                         & 0.07                          \\ \hline
Raiva                 & -0.14                         & -0.08                         \\ \hline
Surpresa              & 0.07                          & 0.24                          \\ \hline
Medo                  & -0.06                         & 0.14                          \\ \hline
\textbf{Felicidade}   & -0.14                         & \textbf{0.31}                 \\ \hline
\textbf{Entropia}     & -0.12                         & \textbf{0.36}                 \\ \hline
\end{tabular}
\end{table}
\end{frame}


\begin{frame}
\frametitle{Slides Adicionais - Prova de Conceito - Discussões}
\begin{block}{Estimular emo\c{c}\~oes diferentes da neutralidade durante a avalia\c{c}\~ao favorece o desempenho dos alunos}
\begin{itemize}
\pause
\item Neutralidade possui uma correla\c{c}\~ao negativa com a propor\c{c}\~ao de acertos
\end{itemize}
\end{block}

\end{frame}

\begin{frame}
\frametitle{Slides Adicionais - Prova de Conceito - Discussões}
\begin{table}[]\footnotesize
\centering
\caption{Resultado​ ​da​ ​correla\c{c}\~ao​ ​de​ ​Pearson​ ​para​ ​cada​ ​emo\c{c}\~ao​ ​detectada
e​ ​a​ ​entropia​ ​contra​ ​os​ ​atributos​ ​das​ ​quest\~oes}
\label{my-label}
\begin{tabular}{|c|c|c|}
\hline
                      & \textbf{Nível de Dificuldade} & \textbf{Proporção de Acertos} \\ \hline
Tristeza	     & -0.33                & 0.27                          \\ \hline
\small \textbf{Neutralidade} & 0.36                 & \small \textbf{-0.48}                \\ \hline
Desprezo     		& -0.15                         & 0.30                 \\ \hline
Desgosto              & -0.13                         & 0.07                          \\ \hline
Raiva                 & -0.14                         & -0.08                         \\ \hline
Surpresa              & 0.07                          & 0.24                          \\ \hline
Medo                  & -0.06                         & 0.14                          \\ \hline
Felicidade   		& -0.14                         & 0.31                 \\ \hline
Entropia     		& -0.12                         & 0.36                 \\ \hline
\end{tabular}
\end{table}
\end{frame}



\begin{frame}
\frametitle{Slides Adicionais - Prova de Conceito - Discussões}
\begin{block}{Estimular emo\c{c}\~oes diferentes da neutralidade durante a avalia\c{c}\~ao favorece o desempenho dos alunos}
\begin{itemize}
\item Neutralidade possui uma correla\c{c}\~ao negativa com a propor\c{c}\~ao de acertos
\pause
\item Felicidade e desprezo possui correla\c{c}\~ao positiva com a propor\c{c}\~ao de acertos
\end{itemize}
\end{block}

\end{frame}

\begin{frame}
\frametitle{Slides Adicionais - Prova de Conceito - Discussões}
\begin{table}[]\footnotesize
\centering
\caption{Resultado​ ​da​ ​correla\c{c}\~ao​ ​de​ ​Pearson​ ​para​ ​cada​ ​emo\c{c}\~ao​ ​detectada
e​ ​a​ ​entropia​ ​contra​ ​os​ ​atributos​ ​das​ ​quest\~oes}
\label{my-label}
\begin{tabular}{|c|c|c|}
\hline
                      & \textbf{Nível de Dificuldade} & \textbf{Proporção de Acertos} \\ \hline
Tristeza	     & -0.33                & 0.27                          \\ \hline
Neutralidade & 0.36                 & -0.48                \\ \hline
\small \textbf{Desprezo}     		& -0.15                         & \small \textbf{0.30}                 \\ \hline
Desgosto              & -0.13                         & 0.07                          \\ \hline
Raiva                 & -0.14                         & -0.08                         \\ \hline
Surpresa              & 0.07                          & 0.24                          \\ \hline
Medo                  & -0.06                         & 0.14                          \\ \hline
\small \textbf{Felicidade}   		& -0.14                         & \small \textbf{0.31}                 \\ \hline
Entropia     		& -0.12                         & 0.36                 \\ \hline
\end{tabular}
\end{table}
\end{frame}




\begin{frame}
\frametitle{Slides Adicionais - Prova de Conceito - Discussões}
\begin{block}{Estimular emo\c{c}\~oes diferentes da neutralidade durante a avalia\c{c}\~ao favorece o desempenho dos alunos}
\begin{itemize}
\item Neutralidade possui uma correla\c{c}\~ao negativa com a propor\c{c}\~ao de acertos
\item Felicidade e desprezo possui correla\c{c}\~ao positiva com a propor\c{c}\~ao de acertos
\pause
\item Entropia das emo\c{c}\~oes possui correla\c{c}\~ao positiva com a propor\c{c}\~ao de acertos
\end{itemize}
\end{block}

\end{frame}

\begin{frame}
\frametitle{Slides Adicionais - Prova de Conceito - Discussões}
\begin{table}[]\footnotesize
\centering
\caption{Resultado​ ​da​ ​correla\c{c}\~ao​ ​de​ ​Pearson​ ​para​ ​cada​ ​emo\c{c}\~ao​ ​detectada
e​ ​a​ ​entropia​ ​contra​ ​os​ ​atributos​ ​das​ ​quest\~oes}
\label{my-label}
\begin{tabular}{|c|c|c|}
\hline
                      & \textbf{Nível de Dificuldade} & \textbf{Proporção de Acertos} \\ \hline
Tristeza	     & -0.33                & 0.27                          \\ \hline
Neutralidade & 0.36                 & -0.48                \\ \hline
Desprezo     		& -0.15                         & 0.30                 \\ \hline
Desgosto              & -0.13                         & 0.07                          \\ \hline
Raiva                 & -0.14                         & -0.08                         \\ \hline
Surpresa              & 0.07                          & 0.24                          \\ \hline
Medo                  & -0.06                         & 0.14                          \\ \hline
Felicidade   		& -0.14                         & 0.31                 \\ \hline
\small \textbf{Entropia}     		& -0.12                         & \small \textbf{0.36}                 \\ \hline
\end{tabular}
\end{table}
\end{frame}

\begin{frame}
\frametitle{Slides Adicionais - Prova de Conceito - Discussões}
\begin{block}{Neutralidade}
\begin{itemize}
\pause
\item A emo\c{c}\~ao mais frequente
\pause
\item Ind\'{i}cio da neutralidade assemelhar com estado de concentra\c{c}\~ao
\pause
\item Neutralidade possui uma correla\c{c}\~ao positiva com quest\~oes dif\'{i}ceis

\end{itemize}
\end{block}

\end{frame}

\begin{frame}
\frametitle{Slides Adicionais - Prova de Conceito - Discussões}
\begin{table}[]\footnotesize
\centering
\caption{Resultado​ ​da​ ​correla\c{c}\~ao​ ​de​ ​Pearson​ ​para​ ​cada​ ​emo\c{c}\~ao​ ​detectada
e​ ​a​ ​entropia​ ​contra​ ​os​ ​atributos​ ​das​ ​quest\~oes}
\label{my-label}
\begin{tabular}{|c|c|c|}
\hline
                      & \textbf{Nível de Dificuldade} & \textbf{Proporção de Acertos} \\ \hline
Tristeza	     & -0.33                & 0.27                          \\ \hline
\small \textbf{Neutralidade} & \small \textbf{0.36}                 & -0.48                \\ \hline
Desprezo     		& -0.15                         & 0.30                 \\ \hline
Desgosto              & -0.13                         & 0.07                          \\ \hline
Raiva                 & -0.14                         & -0.08                         \\ \hline
Surpresa              & 0.07                          & 0.24                          \\ \hline
Medo                  & -0.06                         & 0.14                          \\ \hline
Felicidade   		& -0.14                         & 0.31                 \\ \hline
Entropia     		& -0.12                         & 0.36                 \\ \hline
\end{tabular}
\end{table}
\end{frame}



\begin{frame}
\frametitle{Slides Adicionais - Prova de Conceito - Discussões}
\begin{block}{Neutralidade}
\begin{itemize}
\item A emo\c{c}\~ao mais frequente
\item Ind\'{i}cio da neutralidade assemelhar com estado de concentra\c{c}\~ao
\item Neutralidade possui uma correla\c{c}\~ao positiva com quest\~oes dif\'{i}ceis
\pause
\item Importante identificar os casos diferentes de neutralidade por serem menos frequentes 
e indicar alguma oportunidade de um tutor inteligente atuar
\end{itemize}
\end{block}

\end{frame}
 

 
\end{document}
