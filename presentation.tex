\documentclass{beamer}

\usepackage{bibentry}


\usepackage{natbib}
\setcitestyle{square, comma, numbers,sort&compress, super}
\usepackage{apalike}


\usepackage{color}				% Controle das cores
\usepackage{amsmath}

\usepackage{breqn}
\usepackage{siunitx}
\usepackage{tikz}
\usetikzlibrary{automata,petri,calc,positioning,shapes.geometric,arrows,shapes,backgrounds}


\usepackage[ruled]{algorithm}
\usepackage{algorithmic}
\usepackage{graphicx}

\usepackage[portuges]{babel}    
\usepackage[utf8]{inputenc}

\usepackage{tikz}
\usetikzlibrary{arrows,automata}
\usepackage{subfig}
\usepackage{pdflscape}
\usepackage{pgfgantt}


\usepackage{pdfpages}
\usepackage{siunitx}
\usepackage{multirow}
\usepackage{color}

\newcommand*\rot{\rotatebox{90}}
\bibstyle{apa}

\pgfdeclarelayer{background}
\pgfsetlayers{background,main}

\lecture{Uma Abordagem para Reconhecimento de Emoção por Expressão Facial baseada em Redes Neurais de Convolução}{lecture-text}

%\date{October $31^{th}$, 2017}
\date{Manaus, 30 de Julho de 2018}

% Copyright 2007 by Till Tantau
%
% This file may be distributed and/or modified
%
% 1. under the LaTeX Project Public License and/or
% 2. under the GNU Public License.
%
% See the file doc/licenses/LICENSE for more details.


% Common packages

\usepackage{times}
\mode<article>
{
  \usepackage{times}
  \usepackage{mathptmx}
  \usepackage[left=1.5cm,right=6cm,top=1.5cm,bottom=3cm]{geometry}
}

\usepackage{hyperref}
\usepackage[T1]{fontenc}
\usepackage{tikz}
\usepackage{colortbl}
\usepackage{yfonts}
\usepackage{colortbl}
%\usepackage{translator} % comment this, if not available


% Common settings for all lectures in this course

\def\lecturename{XXVIII Simp\'{o}sio Brasileiro de Inform\'{a}tica na Educa\c{c}\~{a}o}

\title{\insertlecture}

\author{Anderson Cruz, Juan Colonna, Gabriel Leit\~{a}o, \\ Edson Silva, Raimundo Barreto, Tiago Primo} 

\institute
{

\begin{tiny}
 Instituto de Computa\c{c}\~ao (IComp), Centro de Desenvolvimento Tecnol\'{o}gico (CDTec)\\
 %Programa de P\'{o}s-Gradua\c{c}\~{a}o em Inform\'{a}tica, Programa de P\'{o}s-Gradua\c{c}\~{a}o em Computa\c{c}\~ao \\
 Universidade Federal do Amazonas (UFAM), Universidade Federal de Pelotas (UFPEL) \\
 Email: <\{aac, juancolonna, gabriel.leitao, rbarreto\}@icomp.ufam.edu.br, \\ edsonaraujo@ufam.edu.br, tiagoprimo@gmail.com > \\
\end{tiny}
}

\subject{\lecturename}




% Beamer version theme settings
\definecolor{VerdeUFAM}{rgb}{0.06,0.41,0.20}
\useoutertheme[height=0pt,width=2cm,left]{sidebar}
\usecolortheme{rose,sidebartab}
\useinnertheme{circles}
\usefonttheme[only large]{structurebold}

\setbeamercolor{sidebar left}{bg=black!15}
\setbeamercolor{structure}{fg=VerdeUFAM}
\setbeamercolor{author}{parent=structure}

\setbeamerfont{title}{series=\normalfont,size=\LARGE}
\setbeamerfont{title in sidebar}{series=\bfseries}
\setbeamerfont{author in sidebar}{series=\bfseries}
\setbeamerfont*{item}{series=}
\setbeamerfont{frametitle}{size=}
\setbeamerfont{block title}{size=\small}
\setbeamerfont{subtitle}{size=\normalsize,series=\normalfont}

\setbeamertemplate{navigation symbols}{}
\setbeamertemplate{bibliography item}[book]
\setbeamertemplate{sidebar left}
{
  \hbox to2cm{\hss\insertlogo\hss}
%  \vskip1.25em%
%   {\usebeamerfont{title in sidebar}%
% %    \vskip1.0em
%     \hskip1pt
%     \usebeamercolor[fg]{title in sidebar}%
%     \insertshorttitle[width=2cm,left,respectlinebreaks]\par%
%     \vskip1.25em%
%   }%
  \insertverticalnavigation{2cm}%
%  \vfill
  \hbox to 2cm{\hfill\usebeamerfont{subsection in
      sidebar}\strut\usebeamercolor[fg]{subsection in
      sidebar}}
  \vskip1pt%
  {%
    \hskip3pt%
    \usebeamercolor[fg]{author in sidebar}%
    \usebeamerfont{author in sidebar}%
    \insertshortauthor[width=2cm,center,respectlinebreaks]\par%
     \vskip4pt%
  }%
  \hbox to 2cm{
  \usebeamercolor[fg]{author in sidebar}%
    \usebeamerfont{author in sidebar}%
\insertframenumber\hskip5pt}%
}%

\setbeamertemplate{title page}
{
  \vbox{}
  \vskip1em
  {\huge \par}
  {\usebeamercolor[fg]{title}\usebeamerfont{title}\inserttitle\par}%
  \ifx\insertsubtitle\@empty%
  \else%
    \vskip0.25em%
    {\usebeamerfont{subtitle}\usebeamercolor[fg]{subtitle}\insertsubtitle\par}%
  \fi%     
  \vskip1em\par
  \emph{\lecturename}\par On: \insertdate\par
  \vskip0pt plus1filll
  \leftskip=0pt plus1fill\insertauthor\par
  \insertinstitute\vskip1em
}

\logo{\includegraphics[width=2cm]{figuras/brasaocor.jpg}}

% Typesetting Listings

\usepackage{listings}
\lstset{language=Java}

\alt<presentation>
{\lstset{%
  basicstyle=\footnotesize\ttfamily,
  commentstyle=\slshape\color{green!50!black},
  keywordstyle=\bfseries\color{blue!50!black},
  identifierstyle=\color{blue},
  stringstyle=\color{orange},
  escapechar=\#,
  emphstyle=\color{red}}
}
{
  \lstset{%
    basicstyle=\ttfamily,
    keywordstyle=\bfseries,
    commentstyle=\itshape,
    escapechar=\#,
    emphstyle=\bfseries\color{red}
  }
}



% Common theorem-like environments

\theoremstyle{definition}
\newtheorem{exercise}[theorem]{\translate{Exercise}}




% New useful definitions:

\newbox\mytempbox
\newdimen\mytempdimen

\newcommand\includegraphicscopyright[3][]{%
  \leavevmode\vbox{\vskip3pt\raggedright\setbox\mytempbox=\hbox{\includegraphics[#1]{#2}}%
    \mytempdimen=\wd\mytempbox\box\mytempbox\par\vskip1pt%
    \fontsize{3}{3.5}\selectfont{\color{black!25}{\vbox{\hsize=\mytempdimen#3}}}\vskip3pt%
}}

\newenvironment{colortabular}[1]{\medskip\rowcolors[]{1}{blue!20}{blue!10}\tabular{#1}\rowcolor{blue!40}}{\endtabular\medskip}

\def\equad{\leavevmode\hbox{}\quad}

\newenvironment{greencolortabular}[1]
{\medskip\rowcolors[]{1}{green!50!black!20}{green!50!black!10}%
  \tabular{#1}\rowcolor{green!50!black!40}}%
{\endtabular\medskip}




\setbeamertemplate{bibliography item}{\insertbiblabel}
\setbeamercovered{transparent}

\begin{document}

\begin{frame}
 \maketitle
\end{frame}

\begin{frame}{Agenda}
  \tableofcontents
\end{frame}

\section{Introdução}

\begin{frame}
\frametitle{Contexto}
\begin{figure}
\centering
\includegraphics[scale=0.39]{figuras/contexto_1.png}
%\caption{Abordagem Proposta}
\label{fig:problema1}
\end{figure}
\end{frame}

\begin{frame}
\frametitle{Contexto}
\begin{figure}
\centering
\includegraphics[scale=0.39]{figuras/facesbasicas.png}
%\caption{Abordagem Proposta}
\label{fig:problema1}
\end{figure}
\end{frame} 



\begin{frame}
\frametitle{Contexto}
\begin{figure}
\centering
\includegraphics[scale=0.39]{figuras/contexto_21.png}
%\caption{Abordagem Proposta}
\label{fig:problema1}
\end{figure}
\end{frame}


\begin{frame}
\frametitle{Contexto}
\begin{figure}
\centering
\includegraphics[scale=0.39]{figuras/contexto_3.png}
%\caption{Abordagem Proposta}
\label{fig:problema1}
\end{figure}
\end{frame}

\begin{frame}
\frametitle{Motivação}
\begin{figure}
\centering
\includegraphics[scale=0.39]{figuras/contexto_4.png}
%\caption{Abordagem Proposta}
\label{fig:problema1}
\end{figure}
\end{frame}


\begin{frame}
\frametitle{Problema}

\begin{figure}
\centering
\includegraphics[scale=0.39]{figuras/problema_slide_1.png}
%\caption{Abordagem Proposta}
\label{fig:problema1}
\end{figure}

\end{frame}



\begin{frame}
\frametitle{Problema}

\begin{figure}
\centering
\includegraphics[scale=0.22]{figuras/problema_slide_1.png}
%\caption{Abordagem Proposta}
\label{fig:problema1}
\end{figure}

\begin{figure}
\centering
\includegraphics[scale=0.39]{figuras/problema_slide_2.png}
%\caption{Abordagem Proposta}
\label{fig:arquitetura1}
\end{figure}

%%%IMAGEM DO PROBLEMA MUITO DIFICIL ESSA DEFINIÇÃO


\end{frame}


\begin{frame}
\frametitle{Problema}

\begin{figure}
\centering
\includegraphics[scale=0.22]{figuras/problema_slide_1.png}
%\caption{Abordagem Proposta}
\label{fig:problema1}
\end{figure}

\begin{figure}
\centering
\includegraphics[scale=0.39]{figuras/problema_slide_4.png}
%\caption{Abordagem Proposta}
\label{fig:arquitetura1}
\end{figure}
\end{frame}

\begin{frame}
\frametitle{Problema}
\begin{figure}
\centering
\includegraphics[scale=0.39]{figuras/contexto_5.png}
%\caption{Abordagem Proposta}
\label{fig:problema1}
\end{figure}
\end{frame}


%\begin{frame}
%\frametitle{Problema}
%Problemas clássicos:
%\begin{itemize}
%\pause
% \item Ausência de iluminação no ambiente;
% \pause
% \item Rotação do objeto principal, neste caso a face;
% \pause
% \item Escala do objeto principal (face).
%\end{itemize}
%\end{frame}

\begin{frame}
\frametitle{Problema}
\pause
\textit{Como aprimorar os métodos de reconhecimento de emoções por meio da expressão facial a fim de permitir a classificação independente das características do ambiente e de indivíduos para o alcance de maior generalização?} 


\end{frame}


\begin{frame}
\frametitle{Objetivos}
\begin{block}{Objetivo Geral:}
\begin{itemize}
\pause
\item Propor um método para reconhecer emoção humana por expressão facial para classificar emoções básicas em múltiplas faces de uma imagem e comparar a eficácia em cenários de uso real;
\end{itemize}
\end{block}


\frametitle{Objetivos}
\begin{block}{Objetivos Específicos:}
\begin{itemize}
\pause
 \item Propor técnicas de eliminação de ruídos e detecção com recorte das diversas faces de uma imagem;
 \pause
 \item Classificar cada face detectada separadamente estimando a probabilidade para cada emoção básica;
 \pause
 \item Avaliar experimentalmente a solução proposta visando a comparação da eficácia.
\end{itemize}
\end{block}
\end{frame}



\section{Abordagem Proposta}
\begin{frame}
\frametitle{Abordagem Proposta}
\pause
\begin{block}{Monitoramento}
aqui;
\end{block}
\pause
\begin{block}{Pré-Processamento}
aqui;
\end{block}
\pause
\begin{block}{Rede Neural de Convolução}
aqui;
\end{block}
\begin{block}{alguma coisa}
aqui;
\end{block}

\end{frame}


\begin{frame}
\frametitle{Abordagem Proposta - Entrada de Dados }
\begin{figure}
\centering
\includegraphics[scale=0.37]{figuras/arquitetura_1.png}
%\caption{Abordagem Proposta}
\label{fig:arquitetura1}
\end{figure}
\end{frame}

\begin{frame}
\frametitle{Abordagem Proposta - Detecção de Face e Recorte}
\begin{figure}
\centering
\includegraphics[scale=0.27]{figuras/arquitetura_2.png}
%\caption{Abordagem Proposta}
\label{fig:arquitetura2}
\end{figure}
\end{frame}

\begin{frame}
\frametitle{Abordagem Proposta - Pré-Processamento}
\begin{figure}
\centering
\includegraphics[scale=0.37]{figuras/arquitetura_3.png}
%\caption{Abordagem Proposta}
\label{fig:arquitetura3}
\end{figure}
\end{frame}


\begin{frame}
 \frametitle{Abordagem Proposta - Pré-Processamento}

\begin{figure}
\centering
\includegraphics[scale=0.33]{figuras/PreProcessamentoMestrado.png}
%\caption{Alinhamento de Face}
\label{fig:preprocessamento}
\end{figure}

 
\end{frame}


\begin{frame}
\frametitle{Abordagem Proposta - Pré-Processamento}

\pause
\begin{figure}
\centering
\includegraphics[scale=0.23]{figuras/face_alinhada.png}
\caption{Alinhamento de Face}
\label{fig:face_alinhada}
\end{figure}

\pause 
\begin{figure}
\centering
\includegraphics[scale=0.23]{figuras/augmentation.png}
\caption{Aumentação de Dados}
\label{fig:augmentation}
\end{figure}


\end{frame}


\begin{frame}
\frametitle{Abordagem Proposta - Classificação}
\begin{figure}
\centering
\includegraphics[scale=0.33]{figuras/arquitetura_4.png}
%\caption{Abordagem Proposta}
\label{fig:arquitetura4}
\end{figure}
\end{frame}

\begin{frame}
\frametitle{Abordagem Proposta - Visão Geral}
\begin{figure}
\centering
\includegraphics[scale=0.17]{figuras/arquitetura.png}
\caption{Solução Proposta}
\label{fig:arquitetura}
\end{figure}
\end{frame}


%INICIO RESULTADOS E DISCUSSAO
\section{Resultados Parciais}
\begin{frame}
\frametitle{Resultados Parciais}
\begin{figure}
\centering
\includegraphics[scale=0.26]{figuras/tabela_dados.png}
%\caption{Cronograma}
\label{fig:arquitetura}
\end{figure}
\end{frame}



%FIM RESULTADOS E DISCUSSAO


\section{Conclusão}

\begin{frame}
\frametitle{Conclusão}

\end{frame}


\begin{frame}
 \frametitle{Cronograma e Trabalhos futuros}
\begin{figure}
\centering
\includegraphics[scale=0.22]{figuras/cronograma.png}
%\caption{Cronograma}
\label{fig:arquitetura}
\end{figure}
\end{frame}



\begin{frame}{Referencias}
\frametitle{Referências}
    \tiny{\bibliographystyle{apa} }
    \bibliography{base1}
\end{frame}


\section{Agradecimentos}
\begin{frame}{Agradecimentos}
\begin{itemize}
 \item Ao orientador: prof. Barreto; 
\item A banca: prof. Elaine e prof. Daniel;
\item A minha companheira: Giselle;
\item Amigos do grupo de pesquisa;
\item A plateia;
\end{itemize}

\end{frame}

  \begin{frame}{Agradecimentos}
  \begin{center}
 
  \textcolor{VerdeUFAM}{\Large \textbf{Obrigado pela sua atenção!}} \\
  \vspace*{20px}
  \textit{\textbf{Anderson Cruz}} \\
  \textit{\textbf{aac@icomp.ufam.edu.br}}
 
  \end{center}
  \end{frame}

 
 %\begin{frame}
\frametitle{Slides Adicionais - Tabela 1}
\begin{figure}
\centering
\includegraphics[scale=0.45]{figuras/artigosbie.png}
\end{figure}
\end{frame}

\begin{frame}
\frametitle{Slides Adicionais - Tabela 2}
\begin{figure}
\centering
\includegraphics[scale=0.45]{figuras/artigoSBIE2.png}
\end{figure}

\end{frame}


\begin{frame}
\frametitle{Slides Adicionais - Q1}
\begin{figure}
\centering
\includegraphics[scale=0.55]{figuras/q1.png}
\end{figure}
\end{frame}

\begin{frame}
\frametitle{Slides Adicionais - Q2}
\begin{figure}
\centering
\includegraphics[scale=0.5]{figuras/q2.png}
\end{figure}
\end{frame}

\begin{frame}
\frametitle{Slides Adicionais - Q3}
\begin{figure}
\centering
\includegraphics[scale=0.40]{figuras/q3.png}
\end{figure}
\end{frame}

\begin{frame}
\frametitle{Slides Adicionais - Q4}
\begin{figure}
\centering
\includegraphics[scale=0.55]{figuras/q4.png}
\end{figure}
\end{frame}

\begin{frame}
\frametitle{Slides Adicionais - Q5}
\begin{figure}
\centering
\includegraphics[scale=0.45]{figuras/q5.png}
\end{figure}
\end{frame}

\begin{frame}
\frametitle{Slides Adicionais - Q6}
\begin{figure}
\centering
\includegraphics[scale=0.55]{figuras/q6.png}
\end{figure}
\end{frame}

\begin{frame}
\frametitle{Slides Adicionais - Q7}
\begin{figure}
\centering
\includegraphics[scale=0.40]{figuras/q7.png}
\end{figure}
\end{frame}

\begin{frame}
\frametitle{Slides Adicionais - Q8}
\begin{figure}
\centering
\includegraphics[scale=0.45]{figuras/q8.png}
\end{figure}
\end{frame}

\begin{frame}
\frametitle{Slides Adicionais - Q9}
\begin{figure}
\centering
\includegraphics[scale=0.45]{figuras/q9.png}
\end{figure}
\end{frame}

\begin{frame}
\frametitle{Slides Adicionais - Q11}
\begin{figure}
\centering
\includegraphics[scale=0.45]{figuras/q11.png}
\end{figure}
\end{frame}
 
\end{document}
