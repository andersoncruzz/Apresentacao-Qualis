\documentclass{beamer}

\usepackage{bibentry}


\usepackage{natbib}
\setcitestyle{square, comma, numbers,sort&compress, super}
\usepackage{apalike}


\usepackage{color}				% Controle das cores
\usepackage{amsmath}

\usepackage{breqn}
\usepackage{siunitx}
\usepackage{tikz}
\usetikzlibrary{automata,petri,calc,positioning,shapes.geometric,arrows,shapes,backgrounds}


\usepackage[ruled]{algorithm}
\usepackage{algorithmic}
\usepackage{graphicx}

\usepackage[portuges]{babel}    
\usepackage[utf8]{inputenc}

\usepackage{tikz}
\usetikzlibrary{arrows,automata}
\usepackage{subfig}
\usepackage{pdflscape}
\usepackage{pgfgantt}


\usepackage{pdfpages}
\usepackage{siunitx}
\usepackage{multirow}
\usepackage{color}

\usepackage{ragged2e}


\newcommand*\rot{\rotatebox{90}}
\bibstyle{apa}

\pgfdeclarelayer{background}
\pgfsetlayers{background,main}

\lecture{Uma Abordagem para Reconhecimento de Emoção por Expressão Facial baseada em Redes Neurais de Convolução}{lecture-text}

%\date{October $31^{th}$, 2017}
\date{Manaus, 30 de Julho de 2018}

% Copyright 2007 by Till Tantau
%
% This file may be distributed and/or modified
%
% 1. under the LaTeX Project Public License and/or
% 2. under the GNU Public License.
%
% See the file doc/licenses/LICENSE for more details.


% Common packages

\usepackage{times}
\mode<article>
{
  \usepackage{times}
  \usepackage{mathptmx}
  \usepackage[left=1.5cm,right=6cm,top=1.5cm,bottom=3cm]{geometry}
}

\usepackage{hyperref}
\usepackage[T1]{fontenc}
\usepackage{tikz}
\usepackage{colortbl}
\usepackage{yfonts}
\usepackage{colortbl}
%\usepackage{translator} % comment this, if not available


% Common settings for all lectures in this course

\def\lecturename{XXVIII Simp\'{o}sio Brasileiro de Inform\'{a}tica na Educa\c{c}\~{a}o}

\title{\insertlecture}

\author{Anderson Cruz, Juan Colonna, Gabriel Leit\~{a}o, \\ Edson Silva, Raimundo Barreto, Tiago Primo} 

\institute
{

\begin{tiny}
 Instituto de Computa\c{c}\~ao (IComp), Centro de Desenvolvimento Tecnol\'{o}gico (CDTec)\\
 %Programa de P\'{o}s-Gradua\c{c}\~{a}o em Inform\'{a}tica, Programa de P\'{o}s-Gradua\c{c}\~{a}o em Computa\c{c}\~ao \\
 Universidade Federal do Amazonas (UFAM), Universidade Federal de Pelotas (UFPEL) \\
 Email: <\{aac, juancolonna, gabriel.leitao, rbarreto\}@icomp.ufam.edu.br, \\ edsonaraujo@ufam.edu.br, tiagoprimo@gmail.com > \\
\end{tiny}
}

\subject{\lecturename}




% Beamer version theme settings
\definecolor{VerdeUFAM}{rgb}{0.06,0.41,0.20}
\useoutertheme[height=0pt,width=2cm,left]{sidebar}
\usecolortheme{rose,sidebartab}
\useinnertheme{circles}
\usefonttheme[only large]{structurebold}

\setbeamercolor{sidebar left}{bg=black!15}
\setbeamercolor{structure}{fg=VerdeUFAM}
\setbeamercolor{author}{parent=structure}

\setbeamerfont{title}{series=\normalfont,size=\LARGE}
\setbeamerfont{title in sidebar}{series=\bfseries}
\setbeamerfont{author in sidebar}{series=\bfseries}
\setbeamerfont*{item}{series=}
\setbeamerfont{frametitle}{size=}
\setbeamerfont{block title}{size=\small}
\setbeamerfont{subtitle}{size=\normalsize,series=\normalfont}

\setbeamertemplate{navigation symbols}{}
\setbeamertemplate{bibliography item}[book]
\setbeamertemplate{sidebar left}
{
  \hbox to2cm{\hss\insertlogo\hss}
%  \vskip1.25em%
%   {\usebeamerfont{title in sidebar}%
% %    \vskip1.0em
%     \hskip1pt
%     \usebeamercolor[fg]{title in sidebar}%
%     \insertshorttitle[width=2cm,left,respectlinebreaks]\par%
%     \vskip1.25em%
%   }%
  \insertverticalnavigation{2cm}%
%  \vfill
  \hbox to 2cm{\hfill\usebeamerfont{subsection in
      sidebar}\strut\usebeamercolor[fg]{subsection in
      sidebar}}
  \vskip1pt%
  {%
    \hskip3pt%
    \usebeamercolor[fg]{author in sidebar}%
    \usebeamerfont{author in sidebar}%
    \insertshortauthor[width=2cm,center,respectlinebreaks]\par%
     \vskip4pt%
  }%
  \hbox to 2cm{
  \usebeamercolor[fg]{author in sidebar}%
    \usebeamerfont{author in sidebar}%
\insertframenumber\hskip5pt}%
}%

\setbeamertemplate{title page}
{
  \vbox{}
  \vskip1em
  {\huge \par}
  {\usebeamercolor[fg]{title}\usebeamerfont{title}\inserttitle\par}%
  \ifx\insertsubtitle\@empty%
  \else%
    \vskip0.25em%
    {\usebeamerfont{subtitle}\usebeamercolor[fg]{subtitle}\insertsubtitle\par}%
  \fi%     
  \vskip1em\par
  \emph{\lecturename}\par On: \insertdate\par
  \vskip0pt plus1filll
  \leftskip=0pt plus1fill\insertauthor\par
  \insertinstitute\vskip1em
}

\logo{\includegraphics[width=2cm]{figuras/brasaocor.jpg}}

% Typesetting Listings

\usepackage{listings}
\lstset{language=Java}

\alt<presentation>
{\lstset{%
  basicstyle=\footnotesize\ttfamily,
  commentstyle=\slshape\color{green!50!black},
  keywordstyle=\bfseries\color{blue!50!black},
  identifierstyle=\color{blue},
  stringstyle=\color{orange},
  escapechar=\#,
  emphstyle=\color{red}}
}
{
  \lstset{%
    basicstyle=\ttfamily,
    keywordstyle=\bfseries,
    commentstyle=\itshape,
    escapechar=\#,
    emphstyle=\bfseries\color{red}
  }
}



% Common theorem-like environments

\theoremstyle{definition}
\newtheorem{exercise}[theorem]{\translate{Exercise}}




% New useful definitions:

\newbox\mytempbox
\newdimen\mytempdimen

\newcommand\includegraphicscopyright[3][]{%
  \leavevmode\vbox{\vskip3pt\raggedright\setbox\mytempbox=\hbox{\includegraphics[#1]{#2}}%
    \mytempdimen=\wd\mytempbox\box\mytempbox\par\vskip1pt%
    \fontsize{3}{3.5}\selectfont{\color{black!25}{\vbox{\hsize=\mytempdimen#3}}}\vskip3pt%
}}

\newenvironment{colortabular}[1]{\medskip\rowcolors[]{1}{blue!20}{blue!10}\tabular{#1}\rowcolor{blue!40}}{\endtabular\medskip}

\def\equad{\leavevmode\hbox{}\quad}

\newenvironment{greencolortabular}[1]
{\medskip\rowcolors[]{1}{green!50!black!20}{green!50!black!10}%
  \tabular{#1}\rowcolor{green!50!black!40}}%
{\endtabular\medskip}




\setbeamertemplate{bibliography item}{\insertbiblabel}
\setbeamercovered{transparent}

\begin{document}

\begin{frame}
 \maketitle
\end{frame}

\begin{frame}{Agenda}
  \tableofcontents
\end{frame}

\section{Introdução}

\begin{frame}
\frametitle{Contexto}
\begin{figure}
\centering
\includegraphics[scale=0.39]{figuras/contexto_1.png}
%\caption{Abordagem Proposta}
\label{fig:problema1}
\end{figure}
\end{frame}

\begin{frame}
\frametitle{Contexto}
\begin{figure}
\centering
\includegraphics[scale=0.39]{figuras/facesbasicas.png}
%\caption{Abordagem Proposta}
\label{fig:problema1}
\end{figure}
\end{frame} 



\begin{frame}
\frametitle{Contexto}
\begin{figure}
\centering
\includegraphics[scale=0.39]{figuras/contexto_21.png}
%\caption{Abordagem Proposta}
\label{fig:problema1}
\end{figure}
\end{frame}


\begin{frame}
\frametitle{Contexto}
\begin{figure}
\centering
\includegraphics[scale=0.39]{figuras/contexto_3.png}
%\caption{Abordagem Proposta}
\label{fig:problema1}
\end{figure}
\end{frame}

\begin{frame}
\frametitle{Motivação}
\begin{figure}
\centering
\includegraphics[scale=0.39]{figuras/contexto_4.png}
%\caption{Abordagem Proposta}
\label{fig:problema1}
\end{figure}
\end{frame}


\begin{frame}
\frametitle{Problema}

\begin{figure}
\centering
\includegraphics[scale=0.39]{figuras/problema_slide_1.png}
%\caption{Abordagem Proposta}
\label{fig:problema1}
\end{figure}

\end{frame}



\begin{frame}
\frametitle{Problema}

\begin{figure}
\centering
\includegraphics[scale=0.22]{figuras/problema_slide_1.png}
%\caption{Abordagem Proposta}
\label{fig:problema1}
\end{figure}

\begin{figure}
\centering
\includegraphics[scale=0.39]{figuras/problema_slide_2.png}
%\caption{Abordagem Proposta}
\label{fig:arquitetura1}
\end{figure}

%%%IMAGEM DO PROBLEMA MUITO DIFICIL ESSA DEFINIÇÃO


\end{frame}


\begin{frame}
\frametitle{Problema}

\begin{figure}
\centering
\includegraphics[scale=0.22]{figuras/problema_slide_1.png}
%\caption{Abordagem Proposta}
\label{fig:problema1}
\end{figure}

\begin{figure}
\centering
\includegraphics[scale=0.39]{figuras/problema_slide_4.png}
%\caption{Abordagem Proposta}
\label{fig:arquitetura1}
\end{figure}
\end{frame}

\begin{frame}
\frametitle{Problema}
\begin{figure}
\centering
\includegraphics[scale=0.39]{figuras/contexto_5.png}
%\caption{Abordagem Proposta}
\label{fig:problema1}
\end{figure}
\end{frame}


%\begin{frame}
%\frametitle{Problema}
%Problemas clássicos:
%\begin{itemize}
%\pause
% \item Ausência de iluminação no ambiente;
% \pause
% \item Rotação do objeto principal, neste caso a face;
% \pause
% \item Escala do objeto principal (face).
%\end{itemize}
%\end{frame}

\begin{frame}
\frametitle{Problema}
\pause
\textit{Como aprimorar os métodos de reconhecimento de emoções por meio da expressão facial a fim de permitir a classificação independente das características do ambiente e de indivíduos para o alcance de maior generalização?} 


\end{frame}


\begin{frame}
\frametitle{Objetivos}
\begin{block}{Objetivo Geral:}
\begin{itemize}
\pause
\item Propor um método para reconhecer emoção humana por expressão facial para classificar emoções básicas em múltiplas faces de uma imagem e comparar a eficácia em cenários de uso real;
\end{itemize}
\end{block}


\frametitle{Objetivos}
\begin{block}{Objetivos Específicos:}
\begin{itemize}
\pause
 \item Propor técnicas de eliminação de ruídos e detecção com recorte das diversas faces de uma imagem;
 \pause
 \item Classificar cada face detectada separadamente estimando a probabilidade para cada emoção básica;
 \pause
 \item Avaliar experimentalmente a solução proposta visando a comparação da eficácia.
\end{itemize}
\end{block}
\end{frame}



\section{Abordagem Proposta}
%\begin{frame}
%\frametitle{Abordagem Proposta}
%\pause
%\begin{block}{Monitoramento}
%aqui;
%\end{block}
%\pause
%\begin{block}{Pré-Processamento}
%aqui;
%\end{block}
%\pause
%\begin{block}{Rede Neural de Convolução}
%aqui;
%\end{block}
%\begin{block}{alguma coisa}
%aqui;
%\end{block}

%\end{frame}


\begin{frame}
\frametitle{Abordagem Proposta - Monitoramento }
\begin{figure}
\centering
\includegraphics[scale=0.37]{figuras/monitoramento_1.png}
%\caption{Abordagem Proposta}
\label{fig:arquitetura1}
\end{figure}
\end{frame}


\begin{frame}
\frametitle{Abordagem Proposta - Monitoramento }
\begin{figure}
\centering
\includegraphics[scale=0.37]{figuras/monitoramento_2.png}
%\caption{Abordagem Proposta}
\label{fig:arquitetura1}
\end{figure}
\end{frame}

\begin{frame}
\frametitle{Abordagem Proposta - Monitoramento }
\begin{figure}
\centering
\includegraphics[scale=0.37]{figuras/monitoramento_3.png}
%\caption{Abordagem Proposta}
\label{fig:arquitetura1}
\end{figure}
\end{frame}



\begin{frame}
\frametitle{Abordagem Proposta - Detecção de Face e Recorte}
\begin{figure}
\centering
\includegraphics[scale=0.37]{figuras/abordagem_4.png}
%\caption{Abordagem Proposta}
\label{fig:arquitetura2}
\end{figure}
\end{frame}

\begin{frame}
\frametitle{Abordagem Proposta - Pré-Processamento}
\begin{figure}
\centering
\includegraphics[scale=0.34]{figuras/abordagem_5.png}
%\caption{Abordagem Proposta}
\label{fig:arquitetura3}
\end{figure}
\end{frame}

\begin{frame}
\frametitle{Abordagem Proposta - Pré-Processamento}

\pause
\begin{figure}
\centering
\includegraphics[scale=0.23]{figuras/face_alinhada.png}
\caption{Alinhamento de Face}
\label{fig:face_alinhada}
\end{figure}

\pause 
\begin{figure}
\centering
\includegraphics[scale=0.23]{figuras/augmentation.png}
\caption{Aumentação de Dados}
\label{fig:augmentation}
\end{figure}
\end{frame}



\begin{frame}
\frametitle{Abordagem Proposta - Rede Neural de Convolução}
\begin{figure}
\centering
\includegraphics[scale=0.33]{figuras/abordagem_6.png}
%\caption{Abordagem Proposta}
\label{fig:arquitetura3}
\end{figure}
\end{frame}



\section{Prova de Conceito}

\begin{frame}
 \frametitle{Prova de Conceito}
 \begin{block}{Objetivos}
\begin{itemize}
%\justify
\pause
\item Propor um framework para detectar estados emocionais de estudantes baseado em reconhecimento de expressões faciais no contexto das plataformas educacionais;
\pause
\item Monitorar as emoções dos estudantes durante uma avaliação de múltipla escolha (simulado do ENEM);
\pause
\item Correlacionar os estados emocionais detectados e a entropia das emoções com o desempenho durante a avaliação;
\end{itemize}
\end{block}
\end{frame}


\begin{frame}
 \frametitle{Prova de Conceito}
 \begin{table}[]\footnotesize
\centering
\caption{Resultado​ ​da​ ​correlação​ ​de​ ​Pearson​ ​para​ ​cada​ ​emoção​ ​detectada e​ ​a​ ​entropia​ ​contra​ ​os​ ​atributos​ ​das​ ​questões}
\label{my-label}
\begin{tabular}{|c|c|c|}
\hline
                      & \textbf{Nível de Dificuldade} & \textbf{Proporção de Acertos} \\ \hline
\textbf{Tristeza}     & \textbf{-0.33}                & 0.27                          \\ \hline
\textbf{Neutralidade} & \textbf{0.36}                 & \textbf{-0.48}                \\ \hline
\textbf{Desprezo}     & -0.15                         & \textbf{0.30}                 \\ \hline
Desgosto              & -0.13                         & 0.07                          \\ \hline
Raiva                 & -0.14                         & -0.08                         \\ \hline
Surpresa              & 0.07                          & 0.24                          \\ \hline
Medo                  & -0.06                         & 0.14                          \\ \hline
\textbf{Felicidade}   & -0.14                         & \textbf{0.31}                 \\ \hline
\textbf{Entropia}     & -0.12                         & \textbf{0.36}                 \\ \hline
\end{tabular}
\end{table}
 \end{frame}

\begin{frame}
 \frametitle{Prova de Conceito}
 \begin{block}{Discussões}
\begin{itemize}
%\justify
\pause
\item Aqui;

\end{itemize}
\end{block}
\end{frame}
 
 

\section{Experimento}

\begin{frame}
 \frametitle{Experimento}
\begin{block}{Objetivos}
\begin{itemize}
\pause
\item Realizar um estudo comparativo entre AlexNet, Inception-V3 e ResNet;
\pause
\item Avaliar os modelos usando as métricas de precisão, revocação, f1-score e a acurácia;
\pause
\item Utilizar uma base de dados oriunda da natureza e outra do laboratório;
\end{itemize}
\end{block} 
\end{frame}


\begin{frame}
 \frametitle{Experimento}
\begin{block}{Materiais}
\begin{itemize}
\pause
\item \textit{Framework Tensorflow} e \textit{TFLearn};
\pause
\item \textit{OpenCV 3.0};
\pause
\item \textit{GPU NVIDIA GEFORCE 930}, \textit{Intel Core-i7} e \textit{16 GB de RAM DDR4}.

\end{itemize}
\end{block} 
\end{frame}



\begin{frame}
\frametitle{Experimento}
\begin{table}
\tiny
\centering
\caption{As bases de dados foram concatenadas e divididas em três bases: treino, teste e validação. Na seguinte porcentagem: 50\% para treino e 25\% para teste e validação. }
\label{table:basesdivisao}
\begin{tabular}{lcccc}
\hline
\textbf{Base de Dados} & \multicolumn{1}{l}{\textbf{B. de Treino}} & \multicolumn{1}{l}{\textbf{B. de Teste}} & \multicolumn{1}{l}{\textbf{B. de Validação}} & \multicolumn{1}{l}{\textbf{Total de Imagens}} \\ \hline
RAFD                   & 2408                                        & 1206                                       & 1205                                           & 4819                                          \\
CIFE-TRAIN             & 4086                                        & 2042                                       & 2042                                           & 8170                                          \\
CIFE-TEST              & 1759                                        & 879                                        & 878                                            & 3516                                          \\
\textbf{CK}                     & \textbf{1509}                                        & \textbf{754}                                        & \textbf{755}                                            & \textbf{3018}                                          \\
KDEF                   & 1466                                        & 735                                        & 733                                            & 2934                                          \\
JAFFE                  & 105                                         & 53                                         & 55                                             & 213                                           \\
NOVAEMOTIONS           & 16840                                       & 8418                                       & 8417                                           & 33675                                         \\
\textbf{FER}                    & \textbf{11782}                                       & \textbf{5892}                                       & \textbf{5891}                                           & \textbf{23565}                                         \\
Total de Imagens       & 39955                                       & 19979                                      & 19976                                          & 79910                                         \\ \hline
\end{tabular}
\end{table}
\end{frame}

\begin{frame}
\frametitle{Experimento}
\begin{table}[]
\tiny
\centering
\caption{Distribuição das classes (emoções) nas bases de treino, teste e validação. As classes também foram divididas em: 50\% para treino e 25\% para teste e validação.}
\label{table:distclasse}
\begin{tabular}{lcccc}
\hline
\textbf{Classe}  & \multicolumn{1}{l}{\textbf{B. de Treino}} & \multicolumn{1}{l}{\textbf{B. de Teste}} & \multicolumn{1}{l}{\textbf{B. de Validação}} & \multicolumn{1}{l}{\textbf{Total de Imagens}} \\ \hline
Raiva            & 3299                                        & 1650                                       & 1650                                           & 6599                                          \\
Desgosto         & 2453                                        & 1226                                       & 1226                                           & 4905                                          \\
Medo             & 2821                                        & 1411                                       & 1410                                           & 5642                                          \\
Felicidade       & 13943                                       & 6971                                       & 6971                                           & 27885                                         \\
Tristeza         & 4349                                        & 2175                                       & 2174                                           & 8698                                          \\
Surpesa          & 6311                                        & 3156                                       & 3155                                           & 12622                                         \\
Neutralidade     & 6779                                        & 3390                                       & 3390                                           & 13559                                         \\
Total de Imagens & 39955                                       & 19979                                      & 19976                                          & 79910                                         \\ \hline
\end{tabular}
\end{table}
\end{frame}


\begin{frame}
\frametitle{Experimento}
\begin{figure}
\centering
\includegraphics[scale=0.32]{figuras/loss-val.png}
\caption{Função de perda na base de Validação}
\label{fig:arquitetura4}
\end{figure}
\end{frame}

\begin{frame}
\frametitle{Experimento}
\begin{figure}
\centering
\includegraphics[scale=0.32]{figuras/accuracy_val.png}
\caption{Acurácia na base de Validação}
\label{fig:arquitetura4}
\end{figure}
\end{frame}



\section{Resultados Parciais}

\begin{frame}
 \frametitle{Resultados Parciais}
\begin{table}[]
\tiny
\centering
\caption{Resultados experimentais das redes neurais de convolução avaliando a base de validação geral.}
\label{table:resultsexp}
\begin{tabular}{llcccc}
\hline
\textbf{Arquitetura}                & \textbf{Emoção}       & \textbf{Precisão} & \textbf{Revocação} & \textbf{F1-score} & \textbf{Acurácia}               \\ \hline
\multirow{8}{*}{Alexnet}            & Raiva                 & 0.51              & 0.60               & 0.55              & \multirow{8}{*}{0.712}          \\
                                    & Desgosto              & 0.62              & 0.64               & 0.63              &                                 \\
                                    & Medo                  & 0.47              & 0.41               & 0.44              &                                 \\
                                    & Felicidade            & 0.84              & 0.89               & 0.86              &                                 \\
                                    & Tristeza              & 0.64              & 0.50               & 0.56              &                                 \\
                                    & Surpresa              & 0.84              & 0.77               & 0.80              &                                 \\
                                    & Neutralidade          & 0.62              & 0.64               & 0.63              &                                 \\
                                    & Média/Total           & 0.71              & 0.71               & 0.71              &                                 \\ \hline
\multirow{8}{*}{Inception-V3}       & Raiva                 & 0.54              & 0.51               & 0.52              & \multirow{8}{*}{0.701}          \\
                                    & Desgosto              & 0.56              & 0.57               & 0.56              &                                 \\
                                    & Medo                  & 0.47              & 0.42               & 0.44              &                                 \\
                                    & Felicidade            & 0.88              & 0.88               & 0.88              &                                 \\
                                    & Tristeza              & 0.47              & 0.53               & 0.50              &                                 \\
                                    & Surpresa              & 0.85              & 0.79               & 0.82              &                                 \\
                                    & Neutralidade          & 0.59              & 0.62               & 0.61              &                                 \\
                                    & Média/Total           & 0.70              & 0.70               & 0.70              &                                 \\ \hline
\multirow{8}{*}{\textbf{ResNet-34}} & \textbf{Raiva}        & \textbf{0.69}     & \textbf{0.57}      & \textbf{0.62}     & \multirow{8}{*}{\textbf{0.757}} \\
                                    & \textbf{Desgosto}     & \textbf{0.79}     & \textbf{0.66}      & \textbf{0.72}     &                                 \\
                                    & \textbf{Medo}         & \textbf{0.45}     & \textbf{0.50}      & \textbf{0.47}     &                                 \\
                                    & \textbf{Felicidade}   & \textbf{0.90}     & \textbf{0.89}      & \textbf{0.90}     &                                 \\
                                    & \textbf{Tristeza}     & \textbf{0.60}     & \textbf{0.65}      & \textbf{0.63}     &                                 \\
                                    & \textbf{Surpresa}     & \textbf{0.82}     & \textbf{0.86}      & \textbf{0.84}     &                                 \\
                                    & \textbf{Neutralidade} & \textbf{0.67}     & \textbf{0.68}      & \textbf{0.68}     &                                 \\
                                    & \textbf{Média/Total}  & \textbf{0.76}     & \textbf{0.76}      & \textbf{0.76}     &                                 \\ \hline
\end{tabular}
\end{table}
 
\end{frame}



\begin{frame}
 \frametitle{Resultados Parciais}
\begin{table}[]
\tiny
\centering
\caption{Resultados experimentais das redes neurais de convolução avaliando a base de validação CK}
\label{table:ck}
\begin{tabular}{llcccc}
\hline
\textbf{Arquitetura}                   & \textbf{Emoção}       & \multicolumn{1}{l}{\textbf{Precisão}} & \multicolumn{1}{l}{\textbf{Revocação}} & \multicolumn{1}{l}{\textbf{F1-score}} & \multicolumn{1}{l}{\textbf{Acurácia}} \\ \hline
\multirow{8}{*}{Alexnet}         & Raiva                 & 0.91                                  & 1                                      & 0.95                                  & \multirow{8}{*}{0.96}                 \\
                                       & Desgosto              & 0.98                                  & 0.97                                   & 0.98                                  &                                       \\
                                       & Medo                  & 0.89                                  & 0.96                                   & 0.92                                  &                                       \\
                                       & Felicidade            & 0.99                                  & 0.99                                   & 0.99                                  &                                       \\
                                       & Tristeza              & 0.98                                  & 0.84                                   & 0.91                                  &                                       \\
                                       & Surpresa              & 1                                     & 0.94                                   & 0.97                                  &                                       \\
                                       & Neutralidade          & 0                                     & 0                                      & 0                                     &                                       \\
                                       & Média/Total           & 0.97                                  & 0.96                                   & 0.96                                  &                                       \\ \hline
\multirow{8}{*}{Inception-V3}     & Raiva                 & 0.93                                  & 0.94                                   & 0.93                                  & \multirow{8}{*}{0.954}                \\
                                       & Desgosto              & 0.96                                  & 0.94                                   & 0.95                                  &                                       \\
                                       & Medo                  & 0.89                                  & 0.96                                   & 0.92                                  &                                       \\
                                       & Felicidade            & 0.99                                  & 0.98                                   & 0.99                                  &                                       \\
                                       & Tristeza              & 0.91                                  & 0.97                                   & 0.94                                  &                                       \\
                                       & Surpresa              & 1                                     & 0.94                                   & 0.97                                  &                                       \\
                                       & Neutralidade          & 0                                     & 0                                      & 0                                     &                                       \\
                                       & Média/Total           & 0.96                                  & 0.95                                   & 0.96                                  &                                       \\ \hline
\multirow{8}{*}{\textbf{ResNet-34}} & \textbf{Raiva}        & \textbf{0.97}                         & \textbf{0.96}                          & \textbf{0.97}                         & \multirow{8}{*}{\textbf{0.969}}       \\
                                       & \textbf{Desgosto}     & \textbf{1}                            & \textbf{0.92}                          & \textbf{0.96}                         &                                       \\
                                       & \textbf{Medo}         & \textbf{0.91}                         & \textbf{0.99}                          & \textbf{0.95}                         &                                       \\
                                       & \textbf{Felicidade}   & \textbf{0.98}                         & \textbf{0.99}                          & \textbf{0.99}                         &                                       \\
                                       & \textbf{Tristeza}     & \textbf{0.94}                         & \textbf{0.96}                          & \textbf{0.95}                         &                                       \\
                                       & \textbf{Surpresa}     & \textbf{0.98}                         & \textbf{0.99}                          & \textbf{0.99}                         &                                       \\
                                       & \textbf{Neutralidade} & \textbf{0}                            & \textbf{0}                             & \textbf{0}                            &                                       \\
                                       & \textbf{Média/Total}  & \textbf{0.97}                         & \textbf{0.97}                          & \textbf{0.97}                         &                                       \\ \hline
\end{tabular}
\end{table} 
 
\end{frame}



\begin{frame}
 \frametitle{Resultados Parciais}
 \begin{table}[]
 \tiny
\centering
\caption{Resultados experimentais das redes neurais de convolução avaliando a base de validação FER}
\label{table:fer}
\begin{tabular}{llcccc}
\hline
\textbf{Arquitetura}                   & \textbf{Emoção}       & \multicolumn{1}{l}{\textbf{Precisão}} & \multicolumn{1}{l}{\textbf{Revocação}} & \multicolumn{1}{l}{\textbf{F1-score}} & \multicolumn{1}{l}{\textbf{Acurácia}} \\ \hline
\multirow{8}{*}{Alexnet}         & Raiva                 & 0.39                                  & 0.5                                    & 0.44                                  & \multirow{8}{*}{0.543}                \\
                                       & Desgosto              & 0.45                                  & 0.17                                   & 0.25                                  &                                       \\
                                       & Medo                  & 0.37                                  & 0.33                                   & 0.35                                  &                                       \\
                                       & Felicidade            & 0.74                                  & 0.79                                   & 0.76                                  &                                       \\
                                       & Tristeza              & 0.4                                   & 0.29                                   & 0.34                                  &                                       \\
                                       & Surpresa              & 0.72                                  & 0.64                                   & 0.67                                  &                                       \\
                                       & Neutralidade          & 0.49                                  & 0.52                                   & 0.51                                  &                                       \\
                                       & Média/Total           & 0.54                                  & 0.54                                   & 0.54                                  &                                       \\ \hline
\multirow{8}{*}{Inception-V3}     & Raiva                 & 0.43                                  & 0.4                                    & 0.41                                  & \multirow{8}{*}{0.529}                \\
                                       & Desgosto              & 0.13                                  & 0.25                                   & 0.17                                  &                                       \\
                                       & Medo                  & 0.39                                  & 0.36                                   & 0.37                                  &                                       \\
                                       & Felicidade            & 0.81                                  & 0.77                                   & 0.79                                  &                                       \\
                                       & Tristeza              & 0.29                                  & 0.38                                   & 0.32                                  &                                       \\
                                       & Surpresa              & 0.72                                  & 0.64                                   & 0.68                                  &                                       \\
                                       & Neutralidade          & 0.49                                  & 0.46                                   & 0.47                                  &                                       \\
                                       & Média/Total           & 0.55                                  & 0.53                                   & 0.54                                  &                                       \\ \hline
\multirow{8}{*}{\textbf{ResNet-34}} & \textbf{Raiva}        & \textbf{0.61}                         & \textbf{0.42}                          & \textbf{0.5}                          & \multirow{8}{*}{\textbf{0.604}}       \\
                                       & \textbf{Desgosto}     & \textbf{0.69}                         & \textbf{0.28}                          & \textbf{0.39}                         &                                       \\
                                       & \textbf{Medo}         & \textbf{0.35}                         & \textbf{0.47}                          & \textbf{0.4}                          &                                       \\
                                       & \textbf{Felicidade}   & \textbf{0.87}                         & \textbf{0.81}                          & \textbf{0.84}                         &                                       \\
                                       & \textbf{Tristeza}     & \textbf{0.41}                         & \textbf{0.42}                          & \textbf{0.41}                         &                                       \\
                                       & \textbf{Surpresa}     & \textbf{0.71}                         & \textbf{0.76}                          & \textbf{0.73}                         &                                       \\
                                       & \textbf{Neutralidade} & \textbf{0.56}                         & \textbf{0.61}                          & \textbf{0.58}                         &                                       \\
                                       & \textbf{Média/Total}  & \textbf{0.62}                         & \textbf{0.6}                           & \textbf{0.61}                         &                                       \\ \hline
\end{tabular}
\end{table}
\end{frame}




\section{Conclusão}
\begin{frame}
\frametitle{Conclusão}

\end{frame}


\begin{frame}
 \frametitle{Cronograma e Trabalhos futuros}

\begin{table}\footnotesize
\tiny
\caption{Cronograma de Atividades}
\label{table:cronog}
\begin{tabular}{l|c|c|c|c|c|c|c|c|}
\cline{2-9}
                                                                           & \multicolumn{5}{c|}{\textbf{2018}}                                       & \multicolumn{3}{c|}{\textbf{2019}}         \\ \hline
\multicolumn{1}{|m{3cm}|}{\textbf{Atividades}}                                  & \textbf{ago} & \textbf{set} & \textbf{out} & \textbf{nov} & \textbf{dez} & \textbf{jan} & \textbf{fev} & \textbf{mar} \\ \hline
\multicolumn{1}{|m{3cm}|}{Desenvolver e avaliar o componente pré-processamento} & x            &              &              &              &              &              &              &              \\
\hline
\multicolumn{1}{|l|}{Analisar sequência de imagens}                        &              & x            &              &              &              &              &              &              \\
\hline
\multicolumn{1}{|m{3cm}|}{Avaliar experimentalmente outros classificadores}     &              &              & x            &              &              &              &              &              \\
\hline
\multicolumn{1}{|l|}{Implementar e avaliar a MobileNet}                    &              &              &              & x            &              &              &              &              \\
\hline
\multicolumn{1}{|l|}{Avaliar em cenários de uso reais}                     &              &              &              &              & x            & x            &              &              \\
\hline
\multicolumn{1}{|l|}{Escrita da dissertação}                                  &              &              &              & x            & x            & x            & x            & x            \\ \hline
\end{tabular}
\end{table} 
 
\end{frame}



\begin{frame}{Referencias}
\frametitle{Referências}
    \tiny{\bibliographystyle{apa} }
    \bibliography{base1}
\end{frame}


\section{Agradecimentos}
\begin{frame}{Agradecimentos}
\begin{itemize}
 \item Ao orientador: prof. Barreto; 
\item À banca: prof. Elaine e prof. Daniel;
\item À minha companheira: Giselle;
\item Amigos do grupo de pesquisa;
\item À plateia;
\end{itemize}

\end{frame}

  \begin{frame}{Agradecimentos}
  \begin{center}
 
  \textcolor{VerdeUFAM}{\Large \textbf{Obrigado pela sua atenção!}} \\
  \vspace*{20px}
  \textit{\textbf{Anderson Cruz}} \\
  \textit{\textbf{aac@icomp.ufam.edu.br}}
 
  \end{center}
  \end{frame}

 
 %\begin{frame}
\frametitle{Slides Adicionais - Tabela 1}
\begin{figure}
\centering
\includegraphics[scale=0.45]{figuras/artigosbie.png}
\end{figure}
\end{frame}

\begin{frame}
\frametitle{Slides Adicionais - Tabela 2}
\begin{figure}
\centering
\includegraphics[scale=0.45]{figuras/artigoSBIE2.png}
\end{figure}

\end{frame}


\begin{frame}
\frametitle{Slides Adicionais - Q1}
\begin{figure}
\centering
\includegraphics[scale=0.55]{figuras/q1.png}
\end{figure}
\end{frame}

\begin{frame}
\frametitle{Slides Adicionais - Q2}
\begin{figure}
\centering
\includegraphics[scale=0.5]{figuras/q2.png}
\end{figure}
\end{frame}

\begin{frame}
\frametitle{Slides Adicionais - Q3}
\begin{figure}
\centering
\includegraphics[scale=0.40]{figuras/q3.png}
\end{figure}
\end{frame}

\begin{frame}
\frametitle{Slides Adicionais - Q4}
\begin{figure}
\centering
\includegraphics[scale=0.55]{figuras/q4.png}
\end{figure}
\end{frame}

\begin{frame}
\frametitle{Slides Adicionais - Q5}
\begin{figure}
\centering
\includegraphics[scale=0.45]{figuras/q5.png}
\end{figure}
\end{frame}

\begin{frame}
\frametitle{Slides Adicionais - Q6}
\begin{figure}
\centering
\includegraphics[scale=0.55]{figuras/q6.png}
\end{figure}
\end{frame}

\begin{frame}
\frametitle{Slides Adicionais - Q7}
\begin{figure}
\centering
\includegraphics[scale=0.40]{figuras/q7.png}
\end{figure}
\end{frame}

\begin{frame}
\frametitle{Slides Adicionais - Q8}
\begin{figure}
\centering
\includegraphics[scale=0.45]{figuras/q8.png}
\end{figure}
\end{frame}

\begin{frame}
\frametitle{Slides Adicionais - Q9}
\begin{figure}
\centering
\includegraphics[scale=0.45]{figuras/q9.png}
\end{figure}
\end{frame}

\begin{frame}
\frametitle{Slides Adicionais - Q11}
\begin{figure}
\centering
\includegraphics[scale=0.45]{figuras/q11.png}
\end{figure}
\end{frame}
 
\end{document}
