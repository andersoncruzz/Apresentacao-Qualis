%\section{Prova de Conceito}

\begin{frame}
\frametitle{Slides Adicionais - Prova de Conceito - Conceitos}
\begin{block}{Entropia}
Uma medida de dispers\~ao que seu resultado implica que quanto maior a entropia, maior \'e a uniformidade da distribui\c{c}\~ao dos dados 
\end{block}
\begin{figure}
\centering
\includegraphics[scale=0.35]{figuras/entropiaSBIE.png}
\end{figure}
\end{frame}

\begin{frame}
\frametitle{Slides Adicionais - Prova de Conceito - Conceitos}
\begin{block}{Correla\c{c}\~ao de Pearson}
Mede o grau de correla\c{c}\~ao (for\c{c}a e dire\c{c}\~ao) entre um par de vari\'aveis aleat\'orias
\end{block}
\begin{figure}
\centering
\includegraphics[scale=0.4]{figuras/pearson.png}
\end{figure}

\end{frame}



%\begin{frame}
% \frametitle{Prova de Conceito}
% \begin{block}{Objetivos}
%\begin{itemize}
%%%\justify
%\pause
%\item Propor um framework para detectar estados emocionais de estudantes baseado em reconhecimento de expressões faciais no contexto das plataformas educacionais
%\pause
%\item Monitorar as emoções dos estudantes durante uma avaliação de múltipla escolha (simulado do ENEM)
%\pause
%\item Correlacionar os estados emocionais detectados e a entropia das emoções com o desempenho durante a avaliação
%\end{itemize}
%\end{block}
%\end{frame}


\begin{frame}
\frametitle{Slides Adicionais - Prova de Conceito - Discussões}
\begin{table}[]\footnotesize
\centering
\caption{Resultado​ ​da​ ​correla\c{c}\~ao​ ​de​ ​Pearson​ ​para​ ​cada​ ​emo\c{c}\~ao​ ​detectada
e​ ​a​ ​entropia​ ​contra​ ​os​ ​atributos​ ​das​ ​quest\~oes}
\label{my-label}
\begin{tabular}{|c|c|c|}
\hline
                      & \textbf{Nível de Dificuldade} & \textbf{Proporção de Acertos} \\ \hline
\textbf{Tristeza}     & \textbf{-0.33}                & 0.27                          \\ \hline
\textbf{Neutralidade} & \textbf{0.36}                 & \textbf{-0.48}                \\ \hline
\textbf{Desprezo}     & -0.15                         & \textbf{0.30}                 \\ \hline
Desgosto              & -0.13                         & 0.07                          \\ \hline
Raiva                 & -0.14                         & -0.08                         \\ \hline
Surpresa              & 0.07                          & 0.24                          \\ \hline
Medo                  & -0.06                         & 0.14                          \\ \hline
\textbf{Felicidade}   & -0.14                         & \textbf{0.31}                 \\ \hline
\textbf{Entropia}     & -0.12                         & \textbf{0.36}                 \\ \hline
\end{tabular}
\end{table}
\end{frame}


\begin{frame}
\frametitle{Slides Adicionais - Prova de Conceito - Discussões}
\begin{block}{Estimular emo\c{c}\~oes diferentes da neutralidade durante a avalia\c{c}\~ao favorece o desempenho dos alunos}
\begin{itemize}
\pause
\item Neutralidade possui uma correla\c{c}\~ao negativa com a propor\c{c}\~ao de acertos
\end{itemize}
\end{block}

\end{frame}

\begin{frame}
\frametitle{Slides Adicionais - Prova de Conceito - Discussões}
\begin{table}[]\footnotesize
\centering
\caption{Resultado​ ​da​ ​correla\c{c}\~ao​ ​de​ ​Pearson​ ​para​ ​cada​ ​emo\c{c}\~ao​ ​detectada
e​ ​a​ ​entropia​ ​contra​ ​os​ ​atributos​ ​das​ ​quest\~oes}
\label{my-label}
\begin{tabular}{|c|c|c|}
\hline
                      & \textbf{Nível de Dificuldade} & \textbf{Proporção de Acertos} \\ \hline
Tristeza	     & -0.33                & 0.27                          \\ \hline
\small \textbf{Neutralidade} & 0.36                 & \small \textbf{-0.48}                \\ \hline
Desprezo     		& -0.15                         & 0.30                 \\ \hline
Desgosto              & -0.13                         & 0.07                          \\ \hline
Raiva                 & -0.14                         & -0.08                         \\ \hline
Surpresa              & 0.07                          & 0.24                          \\ \hline
Medo                  & -0.06                         & 0.14                          \\ \hline
Felicidade   		& -0.14                         & 0.31                 \\ \hline
Entropia     		& -0.12                         & 0.36                 \\ \hline
\end{tabular}
\end{table}
\end{frame}



\begin{frame}
\frametitle{Slides Adicionais - Prova de Conceito - Discussões}
\begin{block}{Estimular emo\c{c}\~oes diferentes da neutralidade durante a avalia\c{c}\~ao favorece o desempenho dos alunos}
\begin{itemize}
\item Neutralidade possui uma correla\c{c}\~ao negativa com a propor\c{c}\~ao de acertos
\pause
\item Felicidade e desprezo possui correla\c{c}\~ao positiva com a propor\c{c}\~ao de acertos
\end{itemize}
\end{block}

\end{frame}

\begin{frame}
\frametitle{Slides Adicionais - Prova de Conceito - Discussões}
\begin{table}[]\footnotesize
\centering
\caption{Resultado​ ​da​ ​correla\c{c}\~ao​ ​de​ ​Pearson​ ​para​ ​cada​ ​emo\c{c}\~ao​ ​detectada
e​ ​a​ ​entropia​ ​contra​ ​os​ ​atributos​ ​das​ ​quest\~oes}
\label{my-label}
\begin{tabular}{|c|c|c|}
\hline
                      & \textbf{Nível de Dificuldade} & \textbf{Proporção de Acertos} \\ \hline
Tristeza	     & -0.33                & 0.27                          \\ \hline
Neutralidade & 0.36                 & -0.48                \\ \hline
\small \textbf{Desprezo}     		& -0.15                         & \small \textbf{0.30}                 \\ \hline
Desgosto              & -0.13                         & 0.07                          \\ \hline
Raiva                 & -0.14                         & -0.08                         \\ \hline
Surpresa              & 0.07                          & 0.24                          \\ \hline
Medo                  & -0.06                         & 0.14                          \\ \hline
\small \textbf{Felicidade}   		& -0.14                         & \small \textbf{0.31}                 \\ \hline
Entropia     		& -0.12                         & 0.36                 \\ \hline
\end{tabular}
\end{table}
\end{frame}




\begin{frame}
\frametitle{Slides Adicionais - Prova de Conceito - Discussões}
\begin{block}{Estimular emo\c{c}\~oes diferentes da neutralidade durante a avalia\c{c}\~ao favorece o desempenho dos alunos}
\begin{itemize}
\item Neutralidade possui uma correla\c{c}\~ao negativa com a propor\c{c}\~ao de acertos
\item Felicidade e desprezo possui correla\c{c}\~ao positiva com a propor\c{c}\~ao de acertos
\pause
\item Entropia das emo\c{c}\~oes possui correla\c{c}\~ao positiva com a propor\c{c}\~ao de acertos
\end{itemize}
\end{block}

\end{frame}

\begin{frame}
\frametitle{Slides Adicionais - Prova de Conceito - Discussões}
\begin{table}[]\footnotesize
\centering
\caption{Resultado​ ​da​ ​correla\c{c}\~ao​ ​de​ ​Pearson​ ​para​ ​cada​ ​emo\c{c}\~ao​ ​detectada
e​ ​a​ ​entropia​ ​contra​ ​os​ ​atributos​ ​das​ ​quest\~oes}
\label{my-label}
\begin{tabular}{|c|c|c|}
\hline
                      & \textbf{Nível de Dificuldade} & \textbf{Proporção de Acertos} \\ \hline
Tristeza	     & -0.33                & 0.27                          \\ \hline
Neutralidade & 0.36                 & -0.48                \\ \hline
Desprezo     		& -0.15                         & 0.30                 \\ \hline
Desgosto              & -0.13                         & 0.07                          \\ \hline
Raiva                 & -0.14                         & -0.08                         \\ \hline
Surpresa              & 0.07                          & 0.24                          \\ \hline
Medo                  & -0.06                         & 0.14                          \\ \hline
Felicidade   		& -0.14                         & 0.31                 \\ \hline
\small \textbf{Entropia}     		& -0.12                         & \small \textbf{0.36}                 \\ \hline
\end{tabular}
\end{table}
\end{frame}

\begin{frame}
\frametitle{Slides Adicionais - Prova de Conceito - Discussões}
\begin{block}{Neutralidade}
\begin{itemize}
\pause
\item A emo\c{c}\~ao mais frequente
\pause
\item Ind\'{i}cio da neutralidade assemelhar com estado de concentra\c{c}\~ao
\pause
\item Neutralidade possui uma correla\c{c}\~ao positiva com quest\~oes dif\'{i}ceis

\end{itemize}
\end{block}

\end{frame}

\begin{frame}
\frametitle{Slides Adicionais - Prova de Conceito - Discussões}
\begin{table}[]\footnotesize
\centering
\caption{Resultado​ ​da​ ​correla\c{c}\~ao​ ​de​ ​Pearson​ ​para​ ​cada​ ​emo\c{c}\~ao​ ​detectada
e​ ​a​ ​entropia​ ​contra​ ​os​ ​atributos​ ​das​ ​quest\~oes}
\label{my-label}
\begin{tabular}{|c|c|c|}
\hline
                      & \textbf{Nível de Dificuldade} & \textbf{Proporção de Acertos} \\ \hline
Tristeza	     & -0.33                & 0.27                          \\ \hline
\small \textbf{Neutralidade} & \small \textbf{0.36}                 & -0.48                \\ \hline
Desprezo     		& -0.15                         & 0.30                 \\ \hline
Desgosto              & -0.13                         & 0.07                          \\ \hline
Raiva                 & -0.14                         & -0.08                         \\ \hline
Surpresa              & 0.07                          & 0.24                          \\ \hline
Medo                  & -0.06                         & 0.14                          \\ \hline
Felicidade   		& -0.14                         & 0.31                 \\ \hline
Entropia     		& -0.12                         & 0.36                 \\ \hline
\end{tabular}
\end{table}
\end{frame}



\begin{frame}
\frametitle{Slides Adicionais - Prova de Conceito - Discussões}
\begin{block}{Neutralidade}
\begin{itemize}
\item A emo\c{c}\~ao mais frequente
\item Ind\'{i}cio da neutralidade assemelhar com estado de concentra\c{c}\~ao
\item Neutralidade possui uma correla\c{c}\~ao positiva com quest\~oes dif\'{i}ceis
\pause
\item Importante identificar os casos diferentes de neutralidade por serem menos frequentes 
e indicar alguma oportunidade de um tutor inteligente atuar
\end{itemize}
\end{block}

\end{frame}
 
